\section{Economics and Business}

\begin{description}
    \item[GDP (Gross Domestic Product)] \hfill \\ The total monetary value of all final goods and services produced within a country’s borders over a specific period, reflecting the economic output and health of the economy. For instance, if a car is produced in the country and sold for $\mathdollar 20,000$ , that $\mathdollar 20,000$ is its monetary value in GDP, reflecting its price in the market.
    
    \item[Recession] \hfill \\ A recession is a significant, sustained decline in economic activity marked by falling GDP, rising unemployment, and reduced consumer and business spending. We know we are in a recession when there is a significant and sustained decline in economic activity, typically confirmed by Two Consecutive Quarters of Negative GDP Growth (technical definition) or Key indicators like rising unemployment, reduced industrial production, lower consumer spending, and declining business profits
    
    \item[Downturn] \hfill \\ A decline in economic growth. Every recession is a downturn but not every downturn is recession (or indicator of one). It is possible that for a few months, there is a downturn but then economy grows so there is no recession
    \item[Fiscal] \hfill \\ Fiscal usually refers to government finance
    \item[Fiscal stimulus] \hfill \\ Government's policy measures to support economic growth 
    \item[Fiscal Deficit] \hfill \\ A fiscal deficit is the difference between the government's total expenditure (spending on public services, infrastructure, defense, welfare programs, etc.) and its total revenue (mostly tax revenues, like income, corporate, and other taxes). If the government spends more than it earns, it creates a fiscal deficit, which typically leads to the need for borrowing to cover the gap.
    
    \newpage

    \item[Negative Interest Rate] \hfill \\ Used to stimulate economic growth. In a typical interest rate environment, when you deposit money in a bank, you earn interest. However, with negative interest rates, the roles are reversed. If you deposit money in the bank, you might be charged a fee, effectively paying the bank to hold your funds. With negative rates, banks are less likely to hoard cash. They are incentivized to lend more money to businesses and consumers, encouraging economic activity. Negative interest rates make borrowing cheaper for businesses and consumers. When loans are inexpensive, businesses are more likely to invest in projects, expand, or hire workers. Consumers, in turn, are more likely to spend rather than save because saving would lead to a penalty (the bank charges them on their deposits). As spending and investment rise, demand for goods and services increases. Higher demand can push prices upward, which can help prevent deflation (falling prices) and potentially lead to mild inflation.

    \item[Defaulting] \hfill \\ fail to fulfil an obligation, especially to repay a loan or to appear in a law court.

    \item[debt-gdp-ratio] \hfill \\ The debt-to-GDP ratio is the percentage of a country's GDP that is made up of its government debt. A low debt-to-GDP ratio indicates that a country's economy is producing enough goods and services to pay back its debts without taking on more debt. A high debt-to-GDP ratio indicates that a country is less likely to be able to pay back its debt and is at higher risk of defaulting

    \item[Looser fiscal policy] \hfill \\ A looser fiscal policy refers to a government approach that involves increasing spending or reducing taxes (or both) to stimulate economic activity.  By injecting more money into the economy (via spending or tax cuts), the government increases overall demand for goods and services. 

    \item[Inflation ] \hfill \\ In simpler terms, when inflation occurs, you need more money to buy the same amount of goods or services because their prices have increased. Some ways in which inflation can occur is when the demand for goods and services exceeds supply,  cost of production increases (e.g., higher wages or raw material costs), and businesses pass those increased costs onto consumers in the form of higher prices and  when workers demand higher wages because they expect prices to rise (due to previous inflation), and businesses raise prices to cover the cost of higher wages, creating a continuous cycle of inflation. 

\end{description}