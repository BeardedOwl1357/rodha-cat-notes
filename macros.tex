% Writing the expression for finding remainder of fraction numbers
\newcommand{\remFrac}[2]{\displaystyle{ \left. \frac{#1}{#2} \right |_{R} }}

% Referencing to a section with name and section count
% Shows like "section 3.2 Successive Division to find remainder"
% Refer to https://tex.stackexchange.com/a/121871
\newcommand*{\fullref}[1]{ \hyperref[{#1}]{Section \ref*{#1} : \nameref*{#1}}}

\newcommand{\inRange}[2]{
    {#1} \ldots {#2}
}

\newcommand{\bigParen}[1]{
    \left ( #1 \right )
    }

%% #1 = Numerator
%% #2 = Numerator's exponent
%% #3 = Denominator
\newcommand{\eulerRem}[3]{
    \remFrac{#1^{ \remFrac{#2}{E_{#3}} }}{#3}
}

%% Floor and Ceil

\newcommand{\floor}[1]{
    \displaystyle{\left \lfloor {#1} \right \rfloor}
}

\newcommand{\degree}[1]{
    {#1}^{\circ}
}

\newcommand{\Triangle}[1]{
    \Delta {#1}
}

\newcommand{\Area}[1]{
    \text{Area}({#1})
}

\newcommand{\Round}[1]{
    \left [ #1 \right ]
}