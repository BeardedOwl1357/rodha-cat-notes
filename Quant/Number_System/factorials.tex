\section{Factorial}
Factorial of a \textbf{natural number} $n$ is defined as $n! = 1 * 2 * 3 * 4 \ldots n$. Note that $0!$ is 1. For example
\begin{align*}
    5! &= 1 * 2 * 3 * 4 * 5 &= 120 \\
    6! &= 1 * 2 *3 * 4 * 5 * 6 &= 720 \\
\end{align*}

\section{Finding the highest power / index of a number in a factorial}

Suppose that we have a number $15!$ and we need to find the highest power of 2,6 and 8 in the number. Let us take this case-by-case and see what are we doing

\subsection{Successive division}
Before this, we need to understand what successive division is. We divide one number by the other. In the next division, this quotient will be divided by the divisor again. \hl{We keep on doing this until the quotient is less than the number}

Let us successively divide 15 with 2. This is how the table will look like

\begin{table}[ht!]
    \centering
    \begin{tabular}{|| c | c | c ||}
        \hline
        Divisor & Number & Quotient \\
        \hline
        2 & 15 & 7 \\
        2 & 7 & 3 \\
        2 & 3 & 1 \\
        2 & 1 & 0 \\
        \hline
    \end{tabular}
    \caption{Successively dividing 15 with 2}
\end{table}

A short form of the same table (which is mostly used) is as follows. \hl{Note that it is automatically assumed that at each step, the quotient will become the number}

\begin{table}[ht!]
    \centering
    \begin{tabular}{|| c | c ||}
        \hline
        Divisor & Number\\
        \hline
        \textbf{2} & \textbf{15}  \\
        2 & 7   \\
        2 & 3   \\
        2 & 1   \\
        \hline
    \end{tabular}
    \caption{Successively dividing 15 with 2 but short table}
    \label{tab:succ_div_15_2}
\end{table}

\begin{itemize}
    \item The first row represents the initial numbers (divisor and original number). 
    \item The second row is the result of $\displaystyle{ \left \lfloor \frac{15}{2} \right \rfloor} = 7$. This row will indicate that we now need to divide 7 by 2
    \item The third row is the result of $\displaystyle{ \left \lfloor \frac{7}{2} \right \rfloor} = 3$. This row will indicate that we now need to divide 3 by 2. Also, this row is equivalent to dividing 15 by 4
    \item The final row is the result of $\displaystyle{ \left \lfloor \frac{3}{2} \right \rfloor} = 1$. This row will indicate that we now need to divide 1 by 2. Also, this row is equivalent to dividing 15 by 8
\end{itemize}


\subsection{Case 1 : Highest power of 2 in 15!} \label{subsec:highest_power_of_2_in_fact_15}

We will apply successive division between 15 and 2. From \ref{tab:succ_div_15_2}, we can see that the successive division yields 7,3 and 1 respectively. Therefore, the highest degree of 2 in $15!$ is $2^{7 + 3 + 1} \implies 2^{11}$ 

But why?

$15! = 1 * 2 * 3 * 4 * 5 \ldots 15$

From this, we can separate all the factors which are divisible by 2
\begin{align*}
    &= 2 * 4 * 6 * 8 * 10 * 12 * 14 \\
    &= 2 * 2^2 * (2*3) * 2^3 * (2*5) * (2^2 * 3) * (2*7) \\
    \shortintertext{Ignore numbers other than 2. Count number of 2. It will be 11}
\end{align*}

\subsection{Case 2 : Highest power of 6 in 15!}

6 is not a prime number. Every composite number is derived from primes. We need to write the composite number in terms of prime numbers and then find the power of each prime number. 

$6 = 2 * 3$

For 2, we know from section \ref{subsec:highest_power_of_2_in_fact_15} that the highest power of 2 is 11. For 3, we shall find

\begin{table}[ht!]
    \centering
    \begin{tabular}{|| c | c ||}
        \hline
         Divisor & Number  \\
        \hline
         \textbf{3} & \textbf{15} \\
         3 & 5 \\
         3 & 1 \\
        \hline
    \end{tabular}
    \caption{Finding highest power of 3 in $15!$}
\end{table}

The power of 3 is, therefore, $0 + 1 + 5 = 6$.

Now, since 6 is formed by multiplication of 2 and 3, both of them must be in equal numbers. From above, we can see that we have $2^{11}$ and $3^6 \implies$ we have more number of 2 than 3. Since they must be equal, we will take the factor which has the lowest occurrence $\implies 3$

This means that the highest power of 6 in $15!$ is $6^6$

\begin{NOTE}
    Since the highest power of an occurrence is determined by the product of prime factors of the number, instead of finding highest power of each prime factor, we can simply find the highest power of the greatest prime number

    For example, let us say that we want to find the highest power of $30$ in $30!$. $30 = 2 * 3 * 5$. Instead of finding highest power for 2,3 and 5, we could simply find for 5 as the prime factors must be in equal numbers for the composite number to form.
\end{NOTE}


\subsection{Case 3 : Highest power of 8 in 15!}

8 is defined as $2^3$. We can simply find the highest power of 2 and divide that power by 3 (as $8 = 2^3$). From \ref{subsec:highest_power_of_2_in_fact_15}, we know that highest power of 2 is 11. 

Therefore, highest power of 8 in 15! is $2^{\displaystyle{ \left \lfloor \frac{11}{3} \right \rfloor } } = 8^3$


\section{Trailing and Skipping Zeroes}

\subsection{Trailing zeroes in 15!}

Trailing 0 means the number of 0s which are present at the "right" side of the number. For example, in $1050203000000$, there are 6 trailing 0s

\begin{NOTE}
    If you are computer science student, the regex for getting trailing zeroes is 
    \begin{verbatim} 
        ^\d*?(0+)$ 
    \end{verbatim}
\end{NOTE}

To find the number of trailing zeroes in a number, we need to find the highest power of 10. $10 = 2 * 5 \implies$ find highest power of 5

\begin{table}[ht!]
    \centering
    \begin{tabular}{|| c | c ||}
         \hline
         Divisor & Number  \\
         \hline
         \textbf{5} & \textbf{15} \\
         5 & 3 \\
         \hline
    \end{tabular}
    \caption{Finding highest power of 5 in $15!$}
\end{table}

Highest power of $5 = 3 \implies$ highest power of 10 = 3 .

Therefore, the number of trailing zeroes is 3

\newpage

\SampleQuestion{If $2419! = 504^a * b$ where $b$ is not a multiple of 7, then find the value of $a$}

Let us get the prime factors of 504

\begin{table}[ht!]
    \centering
    \begin{tabular}{|| c | c ||}
         \hline
         Divisor & Number  \\
         \hline
         2 & 504 \\
         2 & 252 \\
         2 & 126 \\
         3 & 63 \\
         3 & 21 \\
         7 & 7 \\
           & 1 \\
         \hline
    \end{tabular}
    \caption{Prime factors of 504}
\end{table}

Therefore, $504 = 7 * 8 * 9$

According to question, $2419! = (7 * 8 * 9 ) ^a * b$ where $b$ is not a multiple of 7. Since $b$ cannot be a multiple of 7, we need to ensure that the term $(7*8*9)^a$ must contain all the 7s which are present in $2419! \implies$ we need to find the highest power of 7 in $2419!$

\begin{table}[ht!]
    \centering
    \begin{tabular}{|| c | c ||}
         \hline
         Divisor & Number  \\
        \hline 
         \textbf{7} & \textbf{2419} \\
         7 & 345 \\ 
         7 & 49 \\
         7 & 7 \\
           & 1 \\
        \hline
    \end{tabular}
    \caption{Highest power of 7 in $2419!$}
\end{table}

The highest power of 7 in $2419!$ is $1 + 7 + 49 + 345 = 402$.

This means that $2419! = (7 * 8 * 9)^{402} * b \implies a = 402$

%-----Skipping zeroes

\subsection{Skipping Zeroes}

This is an interesting variation on trailing zeroes. See the below table 

\begin{table}[ht!]
    \centering
    \begin{tabular}{|| c | c ||}
        \hline
         \textbf{Range of factorial} & \textbf{Trailing Zeroes}  \\
        \hline
         0! - 4! & 0 \\
         5! - 9! & 1 \\
         10! - 14! & 2 \\
         15! - 19! & 3 \\
         20! - 24! & 4 \\
         \textbf{25! - 29!} & \textbf{6} \\
         30! - 35! & 7 \\
         45! - 49! & 10 \\
         \textbf{50! - 55!} & \textbf{12} \\
         120! - 124! & 28 \\
         \textbf{125! - 129!} & \textbf{31} \\
        \hline
    \end{tabular}
    \caption{Trailing zeroes of some ranges of factorial}
    \label{tab:trailing_zero_seq}
\end{table}

We can see that till 25!, the number of trailing zeroes is increasing in a linear fashion. However, at 25!, we have two more trailing zeroes than 24!. \hl{This is called a skipping zero}. This is because $25 = 5 * 5$. Since we add two 5s (highest power of 2 is always $\geq$ highest power of 5), we are able to form two pairs of $(5 * 2) \implies $ 100. This will happen at $25n$ factorial (25!,50!,75!,100! etc). \\

Similarly, 124! has 28 zeroes but 125! has 31 zeroes. This is because $125 = 5^3 \implies $ three more pairs of ($5 * 2$) $\implies 1000$. This will happen at $125n$ factorial (125!,250!,375! etc). \hl{This will happen for other powers of 5 as well}

\SampleQuestion{Find the minimum value of $k$ for which no factorial has $k$ trailing zeroes, $k+1$ trailing zeroes or $k+2$ trailing zeroes}

For this to happen, the "skip" should be $\geq 4$, that is, at a certain number $n$, the number of trailing zeroes of $n$ and the number of trailing zeroes of $n+1$ will have a difference of 4. Skipping zeroes are introduced at different powers of 5. To have a skipping zero of $m$, we need $5^m$. \\

In this case, $m = 4 \implies 5^4 = 625$. The trailing zeroes of $625 - 1 = 624$ will be the value of $k$ which will ensure that no factorial has $k,k+1,k+2$ trailing zeroes

\begin{table}[ht!]
    \centering
    \begin{tabular}{|| c | c ||}
         \hline
         Divisor & Number  \\
         \hline
         5 & 624 \\ 
         5 & 124 \\ 
         5 & 24 \\ 
         5 & 4 \\ 
         \hline
    \end{tabular}
    \caption{Highest power of 5 in $624!$}
\end{table}

Therefore, the value of $k$ is $k = 4 + 24 + 124 = 152$

\section{Some questions}

\SampleQuestion{Find the number of times 49 is present in $50! - 49!$}

We can simplify the expression $50! - 49!$ as $49! (50-1) \implies 49! * 49$. We now need to find how many times 49 is present in 49! (highest power of $49 = 7 * 7 \impliedby 7^2$) and then, add 1 to it

\begin{table}[ht!]
    \centering
    \begin{tabular}{|| c | c ||}
         \hline
         Divisor & Number  \\
         \hline
         \textbf{7} & \textbf{49} \\ 
         7 & 7 \\ 
           & 1 \\ 
         \hline
    \end{tabular}
    \caption{Highest power of 7 in $49!$}
\end{table}

Highest power of 7 in 49! = $7^8 \implies$ highest power of 49 = $7^{\frac{8}{2}} = 7^4 = 4$ times.

In the expression, we have a "49" with us $\implies$ 49 will be present 5 times

\SampleQuestion{Find the number of times 98 is present in $99! - 98!$}

We can simplify the expression $99! - 98!$ as $98! (99-1) \implies 98! * 98$. We now need to find how many times 98 is present in 98! (highest power of $98 = 2 * 7 * 7 \implies 7^2$) and then, add 1 to it

\newpage

\begin{table}[ht!]
    \centering
    \begin{tabular}{|| c | c ||}
         \hline
         Divisor & Number  \\
         \hline
         \textbf{7} & \textbf{98} \\ 
         7 & 14 \\ 
           & 2 \\ 
         \hline
    \end{tabular}
    \caption{Highest power of 7 in $98!$}
\end{table}

As we can see in the table, Highest power of 7 in 98! = $7^16 \implies$ highest power of 49 = $7^{\frac{16}{2}} = 7^8 = 8$ times. Since the highest of power of 2 will be greater than highest power of $7^2$, we are not counting for 2.

In the expression, we have a "98" with us $\implies$ 98 will be present 9 times

\SampleQuestion{Find number of trailing zeroes in the expression $1! * 2! * 3! * \ldots 30! $}

From table \ref{tab:trailing_zero_seq}, we can see the number of trailing zeroes for some sequence. To get the final result, we will multiply and add

\begin{align*}
    &= (0 * 4) + (1 * 5) + (2 * 5) + (3 * 5) + (4 * 5) + (6 * 5) + 7 \\
    &= 5 * (1 + 2 + 3 + 4 + 6) + 7 \\
    &= 87
\end{align*}

\SampleQuestion{Find the highest power of 5 in the product of first 50 multiples of 5}

The first 50 multiples of 5 can be written as

\begin{align*}
    &= (5 * 1) * (5 * 2) * (5 * 3) \ldots (5 * 50) \\
    &= 5^{50} (1 * 2 * 3 * 4 \ldots 50) \\
    &= 5^{50} * (50!)
\end{align*}

Now we need to find the highest power of 5 in 50!. Referring to \ref{tab:trailing_zero_seq}, we get that it is 12. Therefore, in the product of first 50 multiples of 50, the highest power of 5 will be $50 + 12 = 62$ 

\SampleQuestion{How many natural numbers less than 200 are there such that $(n-1)!$ is not divisible by $n$}

\hl{This is a good question}. First, we need to plug in some values and see if we find a pattern or not \newpage

\begin{table}[ht!]
    \centering
    \begin{tabular}{|| c | c | c ||}
         \hline
         $(n-1)!$ & $n$ & $(n-1)!$ Divisible by $n$ ?  \\
         \hline
         0! & 1 & Yes \\
         1! & 2 & No \\
         2! & 3 & No \\
         3! & 4 & No \\
         4! & 5 & No \\
         5! & 6 & Yes \\
         6! & 7 & No \\
         7! & 8 & Yes \\
         8! & 9 & Yes \\
         \hline
    \end{tabular}
    \caption{Some natural numbers $<$ 200 are there such that $(n-1)!$ is not divisible by $n$}
\end{table}

We can observe in the table that \textbf{except $n=4$}, $(n-1)!$ is not divisible by $n$ when $n$ is a prime number. Therefore, the number of natural numbers which are less than 200 where $(n-1)!$ is not divisible by $n$ is \textbf{count of prime numbers between 1-200} + 1 (as 4 is a composite number which satisfies this condition) \\

Number of primes between 1-100 = 25, 101-200 = 21 $\implies 25 + 21 + 1 = 47$

\SampleQuestion{How many values of $p$, which is a prime number, are present where the expression $\frac{10000!}{p^p}$ is an integer $\implies p^p$ divides 10000! ?}

\hl{This is a good question}. $10000!$ will be divisible by $p^p$ as long as the highest power of $p$ in 10000! $\leq p$. We need to find a value of $p$ where highest power of $p$ in 10000! $\ge p$.  

$10000 = 100 * 100$. The next prime value after 97 is 101, which is greater than 100. 10000 is a perfect square but 10000 is a composite number so we need to find the nearest prime number to 100 and check for which prime number, the highest power is lesser than itself. 

Let us do a quick test for 97 and 101
\begin{itemize}
    \item $97; 10000 = (97 * 103) + 9 ; 103 = (97 * 1) + 6; highest power = 104$
    \item $101; 10000 = (101 * 90) + 9100 ; highset power = 90$
\end{itemize}

We can see that 101 has a power which is less than "101" $\implies p \le 101$ for the condition to hold true. \textbf{The number of primes between 1-100 is 25}.

\SampleQuestion{Find the value of tens digit of $N = 1! + 2! +3! + 4! + \ldots 100!$}

From table \ref{tab:trailing_zero_seq}, we can see that after $10!$, the number of trailing zeroes $\geq 2 \implies$ they will not have any effect on the tens place of the final result. The tens digit can be determined as 

$1 + 2 + 6 + 24 + 120 + 720 + xx40 + xxx20 + xxxx80$ (here, $x$ means that we don't care about that digit as we are concerned with tens place)

$9 + 24 + 20 + 20 + 40 + 20 + 80 = x13$. Tens digit = 1

\subsection{Questions where we know the highest power of a number and need to approximate the value of factorial}

In earlier questions, we have dealt with finding the highest power of a number in $n!$. In the questions below, we will be given the highest power of a number and we would need to find the value of $n$ where $n!$ contains the highest power of the given number

\begin{NOTE}
    As mentioned in a comment on youtube, \href{https://www.youtube.com/watch?v=sbyU48ZOsVM&list=PLG4bwc5fquzgfMh4YFDnv7fttM0RIKiUQ&index=8}{To all those who are wondering, always multiply the power of prime no by prime no-1, you'll get the approx idea of where the exact no will lie} \\

    For example, if we want to know the value of $n$ for which $n!$ will have highest power of 5 as 15, we can calculate $15 * ( 5 - 1 ) = 60$. Breaking 60 into factors.

    \begin{align*}
        60 &= 5 * \textbf{12} + 0 \\
        12 &= 5 * \textbf{2} + 2 \\
    \end{align*}

    Sum = 2 + 12 = 14

    Using this, we get an approximate value of $n$ for which highest power of 5 is near 15. If $60!$ has highest power of 5 as 14, then definitely $65!$ will have highest power of 5 as 15 \\

    \textbf{This is more of a result / approximation that you can use}. I don't know the exact derivation behind it, however chatgpt says that in Legendere's Formula : Exponent of prime $P$ in $N! = \floor{\dfrac{N}{P}} + \floor{\dfrac{N}{P^2}} + \floor{\dfrac{N}{P^3}} + \ldots$. In this, $\floor{\dfrac{N}{P}}$ is the dominant term however other terms do have an impact as well. We assume that $\floor{\dfrac{N}{P}} = M$ however $P * M$ is not correct as this overestimates the actual position because it does not account for the other terms. So, we use $(P-1) * M$ to get a starting value to look for the exact term where $P^M$ is present in $N!$
\end{NOTE}

\SampleQuestion{How many values can $N$ take if $N!$ is a multiple of $3^{20}$ but not $3^{25}$ ?}

An approximate value of $n$ where highest power of 3 can be around 20 is $20 * (3-1) = 40$. Let us find the highest power of 3 in $40!$

\begin{table}[ht!]
    \centering
    \begin{tabular}{|| c | c ||}
         \textbf{3} & \textbf{40}  \\
         \hline 
         3 & 13 \\
         3 & 4 \\
         3 & 1 \\
    \end{tabular}
    \caption{Highest power of 3 in $40!$}
\end{table}

Sum = 1 + 4 + 13 = 18

42 will have highest power of 3 as 19 (as $42 = 3 * 14$) and 45 will have highest power of 3 as 21 ($45 = 9 * 5 \implies 3^2 * 5$). Since $3^{21} = 3^{20} * 3$, we can say that $45!$ can be used as our starting value. \hl{We now need to find a number which is not a multiple of $3^{25}$}    

Let us find the highest power of 3 in $( 25 * (3-1))! \implies 50!$. 

\begin{table}[ht!]
    \centering
    \begin{tabular}{|| c | c ||}
         \textbf{3} & \textbf{50}  \\
         \hline 
         3 & 16 \\
         3 & 5 \\
         3 & 1 \\
    \end{tabular}
    \caption{Highest power of 3 in $50!$}
\end{table}

Sum = 1 + 5 + 16 = 22

\begin{itemize}
    \item $51!$ will have highest power of 3 as 23 as $51 = 3 * 17$
    \item $54!$ will have highest power of 3 as 26 as $54 = 9 * 6 \implies 3^3 * 2$. 
\end{itemize}

Based on above, the number of values which are divisible by $3^{20}$ but not by $3^{25}$ are in the range $\displaystyle \left [ 45!,53! \right ] \implies 53 - 45 + 1 = 9$. Refer to section \ref{sub-two-nums} 

\SampleQuestion{How many values can $n$ take if $n!$ is a multiple of $2^{25}$ but not $3^{25}$ ?}

\textbf{Finding the value of $n$ for which $2^{25}$ is a multiple}

Approx value of $n$ = $25 * (2 - 1) = 25$.

Finding highest power of 2 in 25
\begin{align*}
    25 &= 2 * \textbf{12} + 1 \\
    12 &= 2 * \textbf{6} + 0 \\
    6 &= 2 * \textbf{3} + 0 \\
    3 &= 2 * \textbf{1} + 0 \\
\end{align*}

Sum = 1 + 3 + 6 + 12 = 22

Following this, $26!$ will have $2^{23}$, \textbf{$28!$} will have $2^{25}$ (as $28 = 4 * 7 \implies 2^2 * 7$). \\

\textbf{Finding the value of $n$ for which $3^{25}$ is a multiple}

Approx value of $n$ = $25 * (3-1) = 50$

Highest power of 3 in 50!

\begin{align*}
    50 &= 3 * \textbf{16} + 2 \\
    16 &= 3 * \textbf{5} + 1 \\
    5 &= 3 * \textbf{1} + 2 \\
\end{align*}

SUM = 1 + 5 + 16 = 22

$51!$ will have $3^{23}$ and \textbf{$54!$} will have $3^{26}$ (as $54 = 9*6 \implies 3^3 * 2$). 

Therefore, the number of values which $n$ can take to ensure that it is a multiple of $2^{25}$ but not $3^{25}$ is $53 - 28 + 1 = 26$

\SampleQuestion{Factorial of a number $n$ is having 95 trailing zeroes. What could be the least value of $n$ ?}

Since we have 95 trailing zeroes, we can say that the highest power of 5 is 95 (as the product of 2 and 5 determines the number of trailing zeroes). The approximate number which has highest power of 5 as 95 is $95 * (5 - 1) = 380$. \\ 

Highest power of 5 in $380!$ : 

\begin{align*}
    380 &= 5 * \textbf{76} + 0 \\
    76 &= 5 *  \textbf{15} + 1 \\
    15 &= 5 *  \textbf{3} + 0 \\
\end{align*}

Sum = 3 + 15 + 76 = 94

This means that $385!$ will have highest power of 5 as 95 $\implies n = 385$

\SampleQuestion{Factorial of a number $n$ is having 62 trailing zeroes. What could be the least value of $n$ ?}

Approx value of $n$ = $62 * (5-1) = 248$. Highest power of 5 in $248!$ is 

\begin{align*}
    248 &= 5 * \textbf{49} + 3 \\
    49 &= 5 *  \textbf{9} + 4 \\
    9 &= 5 *   \textbf{1} + 4
\end{align*}

Sum = 1 + 9 + 49 = 59

$250!$ will have highest power of 5 as 62 (As $250 = 125 * 2 \implies 5^3 * 2$). Therefore, $n = 250$

\section{Rightmost Non Zero Digit of factorial of a number}

This is a rather obscure topic and is mostly formula based. The approach is defined in the below table

\begin{table}[ht!]
    \centering
    \begin{tabular}{|| c | c | c ||}
         \hline
         $n$ & Write $n$ in form & Non Zero Rightmost Digit  \\
         \hline
         $< 25$ & $5a + b$ & $2^a a! b!$ \\
         $\leq 25 \leq 100$ & $25a + b$ & $4^a a! b!$ \\
         \hline
    \end{tabular}
    \caption{Formulate for Rightmost Non Zero Digit of factorial of a number}
\end{table}

\textbf{Note that while calculating the value of rightmost digit using the formula above, just calculate the first digits and then multiply them}. So, for example, instead of calculating $a!$ completely, only calculate its first digit and use it in subsequent multiplication

\SampleQuestion{Find rightmost non zero digit of 19!}

$19 = 5*3 + 4 \implies a=3,b=4$

The rightmost non-zero digit is 
\begin{align*}
    &= 2^3 * 3! * 4! &\text{$4! = 24$ but since we are only concerned with the "ones" digit, we can use 4} \\
    &= 8 * 6 * 4 &\text{($8 * 6 * 4 = 48 * 4 \implies 192$. We only care about the "ones" digit so "2")} \\
    &= 2 \\ 
\end{align*}

\SampleQuestion{Find rightmost non zero digit of 77!}

$77 = 25 * 3 + 2 \implies a = 3, b = 2$

The rightmost non-zero digit is 
\begin{align*}
    &= 4^3 * 3! * 2! &\text{($4^3 = 64$. We are only concerned with ones digit)} \\
    &= 4 * 3 * 2 \\
    &= 4 &\text{($4*3*2 = 24$ but we only care about ones digit)} \\
\end{align*}

\SampleQuestion{Find rightmost non zero digit of 99!}

Let us use some logic. $100! = 100 * 99!$. In this specific case, we can try finding the rightmost non-zero digit for 100 and then conclude that it is the same digit for $99$ as well as $100!$ is just $99! * 100$. By multiplying 100, we are not changing the rightmost non-zero digit of the number

\begin{align*}
    &= 100 = 25 * 4 + 0 \\
    \text{Rightmost non-zero digit} &= 4^{4} * 4! * 0! \\
    &= 6 * 4 * 1 &\text{($4^4 = 256, 4! = 24$. Consider only ones digit)} \\
    &= 4
\end{align*}

\section{Rightmost last two non-zero digits of n!}

Write $n$ in the form of $5a + b$. The formula for the rightmost non-zero two digits is $\displaystyle{\left ( 12^a * a! * \prod_{k=1} ^{b} (5a+k) \right )}$

\SampleQuestion{Find last 2 non-zero digits of $21!$}
\begin{align*}
    21 &= 5*4 + 1 \implies a=4,b=1 \\
    \text{Rightmost non-zero digit of $21!$} &= 12^4 * 4! * (5a + 1) \\
    &= 36 * 24 * 21 &\text{$12^4 = 20736$} \\
    &= 44
\end{align*}

\section{Numbers which are sum of factorials of their digits}

This is more like a memorization thing. They can ask question about "numbers which are sum of factorials of their digits" and since there are only a handful of such numbers, we can remember them and quickly solve the question. The list of such numbers are as follows

\begin{table}[ht!]
    \centering
    \begin{tabular}{|| c | c ||}
         \hline
         Number & Expansion  \\
         \hline
         $1!$ & $1$ \\ 
         $2!$ & $2$ \\ 
         $145!$ & $1! + 4! + 5!$ \\ 
         $40585!$ & $4! + 0! + 5! + 8! + 5!$ \\ 
         \hline
    \end{tabular}
    \caption{Numbers which are sum of factorials of their digits}
    \label{table:num-sum-fact-digit}
\end{table}

\SampleQuestion{If $A! + B! + C! = ABC$, where $ABC$ is a three digit number, find the highest power of 7 in $ABC!$}

As we can see in table \ref{table:num-sum-fact-digit}, there is only one 3 digit number which satisfies the above property $\implies$ 145. We now need to find the highest power of 7 in 145!

\begin{align*}
    145 &= 7 * \textbf{20} + 5 \\
    20 &= 7 * \textbf{2} + 6 \\
\end{align*}

Highest power of 7 = $2 + 20 \implies 7^{22}$

\section{Digits in n!}

\begin{table}[ht!]
    \centering
    \begin{tabular}{|| c | c ||}
        \hline
        \textbf{Condition} & \textbf{Applicable values}  \\
        \hline
        $n!$ has $n$ digits & $n \in [1,22,23,24]$ \\ 
        $n!$ has $< n$ digits & $2 \leq n \leq 21$ \\ 
        $n!$ has $> n$ digits & $n \geq 25$ and $n=0$ \\ 
        \hline
    \end{tabular}
    \caption{Digits in $n!$}
    \label{tab:n-and-num-of-digits}
\end{table}

\SampleQuestion{If $n!$ has $n$ digits, then there would be how many zeroes at the end of factorial of sum of all such values of $n$ ?}

Taking reference from \ref{tab:n-and-num-of-digits}, we can see that $n \in [1,22,23,24] \implies (1+22+23+24) = 70$. We need to find the number of trailing zeroes on $70! \implies$ find highest power of 5 in 70!

\begin{align*}
    70 &= 5 * \textbf{14} + 0 \\
    14 &= 5 * \textbf{2} + 4 \\
\end{align*}

$14 + 2 = 16$ trailing zeroes in $70!$

\newpage
\section{Miscellaneous Questions}

\SampleQuestion{How many 2 digit numbers are there in which product of factorials of its digits $ > $ sum of factorials of digits?}

Since we are dealing with 2 digit numbers, we need to check on the range $[10,99]$. What we effectively need to check is for a number $AB$, which values of $A$ and $B$ satisfy $A! * B! > A! + B!$. Let us try with some values

\begin{table}[ht!]
    \centering
    \begin{tabular}{|| c | c | c | c | c ||}
         \hline
         $A$ & $B$ & $AB$ & $A! * B! > A! + B!$ & Useful?  \\
         \hline
         1 & 0 & 10 & $1! * 0! < 1! + 0!$       & N \\
         1 & 1 & 11 & $1! * 1! < 1! + 1!$       & N  \\
         1 & 2 & 12 & $1! * 2! < 1! + 2!$       & N  \\
         1 & 3 & 13 & $1! * 3! < 1! + 3!$       & N  \\
         1 & 4 & 14 & $1! * 4! < 1! + 4!$       & N  \\
         1 & 5 & 15 & $1! * 5! > 1! + 5!$       & N  \\
         2 & 0 & 20 & $2! * 0! < 2! + 0!$       & N \\
         2 & 1 & 21 & $2! * 1! < 2! + 1!$       & N  \\
         2 & 2 & 22 & \hl{$2! * 2! = 2! + 2!$}  & N  \\
         2 & 3 & 23 & $2! * 3! > 2! + 3!$       & Y  \\
         3 & 0 & 30 & $3! * 0! < 3! + 0!$       & N \\
         3 & 1 & 31 & $3! * 1! < 3! + 1!$       & N  \\
         3 & 2 & 32 & $3! * 2! > 3! + 2!$       & Y  \\
         3 & 3 & 33 & $3! * 3! > 3! + 3!$       & Y  \\
         4 & 0 & 40 & $4! * 0! < 4! + 0!$       & N \\
         4 & 1 & 41 & $4! * 1! < 4! + 1!$       & N  \\
         4 & 2 & 42 & $4! * 2! > 4! + 2!$       & Y  \\
         4 & 3 & 43 & $4! * 3! > 4! + 3!$       & Y  \\
         \hline
    \end{tabular}
    \caption{A table with sample values for $A! * B! > A! + B!$}
\end{table}

\begin{itemize}
    \item We can see that for any number of the form $AB$ where $A=1$ and $B \in [0,9]$, all numbers will have sum of factorial of digits $ > $ product of factorial of digits

    \item In the range $20 - 29$, after $23$, the product of factorial of digits > sum of factorial of digits $\implies (29 - 23 + 1) = \textbf{7}$ numbers satisfy the property. 
    
    \item In the range $30 - 39$, after $32$, the product of factorial of digits > sum of factorial of digits $\implies (39 - 32 + 1) = \textbf{8}$ numbers satisfy the property.  

    \item In the range $40 - 49$, after $42$, the product of factorial of digits > sum of factorial of digits $\implies (49 - 42 + 1) = \textbf{8}$ numbers satisfy the property. \hl{This will be true for all the upcoming numbers}  
\end{itemize}

Total count = $7 + 8 * (7) = 63$

\vspace{2cm}

\SampleQuestion{How many 2 digit numbers exceed the sum of factorials of their digits?}

According to question, a 2 digit number $AB$ where $A and B are digits where A \in [1,9] and B \in [0,9]$. Since we are dealing with only 2 digit numbers, we are confined to the range $[10,99]$. We need to find the count of 2 digit numbers where $AB > A! + B!$. Let us plug some values and see what happens

\begin{table}[ht!]
    \centering
    \begin{tabular}{|| c | c | c | c | c ||}
        \hline
         \textbf{$A$} & \textbf{$B$} & \textbf{$AB$} & \textbf{$A! + B!$} & \textbf{$AB! > A! + B!$}  \\
        \hline
         1 & 0 & 10 & $1! + 0! = 2$ & Y \\
         1 & 1 & 11 & $1! + 1! = 2$ & Y \\
         1 & 2 & 12 & $1! + 2! = 3$ & Y \\
         1 & 3 & 13 & $1! + 3! = 7$ & Y \\
         1 & 4 & 14 & $1! + 4! = 25$ & N \\
         1 & 5 & 15 & $1! + 5! = 121$ & N \\
         2 & 0 & 20 & $2! + 0! = 3$ & Y \\
         2 & 1 & 21 & $2! + 1! = 3$ & Y \\
         2 & 2 & 22 & $2! + 2! = 4$ & Y \\
         2 & 3 & 23 & $2! + 3! = 8$ & Y \\
         2 & 4 & 24 & $2! + 4! = 26$ & N \\
         2 & 5 & 25 & $2! + 5! = 122$ & N \\
         3 & 0 & 30 & $3! + 0! = 4$ & Y \\
         3 & 1 & 31 & $3! + 1! = 4$ & Y \\
         3 & 2 & 32 & $3! + 2! = 5$ & Y \\
         3 & 3 & 33 & $3! + 3! = 12$ & Y \\
         3 & 4 & 34 & $3! + 4! = 30$ & Y \\
         3 & 5 & 35 & $3! + 5! = 126$ & N \\
         4 & 0 & 40 & $4! + 0! = 25$ & Y \\
         4 & 1 & 41 & $4! + 1! = 25$ & Y \\
         4 & 2 & 42 & $4! + 2! = 26$ & Y \\
         4 & 3 & 43 & $4! + 3! = 32$ & Y \\
         4 & 4 & 44 & $4! + 4! = 48$ & N \\
         5 & 0 & 50 & $5! + 0! = 121$ & N \\
        \hline
    \end{tabular}
    \caption{Checking whether $AB! > A! + B!$ for some 2 digit numbers}
\end{table}

We can observe that once any 2 digit number has the digit 5 in it, the sum of the factorial of digits definitely increases as $5! = 120 \implies $ greater than any 2 digit numbers. Therefore

\begin{itemize}
    \item For the range 10-20, numbers in range $[10,14]$ \hl{can be} such that $AB > A! + B! \implies [10,11,12,13]$
    \item For the range 20-30, numbers in range $[20,24]$ \hl{can be} such that $AB > A! + B! \implies [20,21,22,23]$
    \item For the range 30-40, numbers in range $[30,34]$ \hl{can be} such that $AB > A! + B! \implies [30,31,32,33,34]$
    \item For the range 40-50, numbers in range $[40,44]$ \hl{can be} such that $AB > A! + B! \implies [40,41,42,43]$ 
\end{itemize}

Based on the above table, we can find the count : $(4 + 4 + 5 + 4) = 17$.