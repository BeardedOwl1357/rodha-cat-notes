\section{General form of number and common divisibility rules}

Any number of the form $abc$ can be written in base10 as $( 10^2 * a ) + ( 10^1 * b ) + ( 10^0 * c )$. For example, we can write 567 as $500 + 60 + 7$. Similar behavior can be done for any number with any digits. In general, a number $a_1 a_2 a_3 \ldots a_n$ can be written as $(10^{n-1} * a_1) + (10^{n-2} * a_2) + (10^{n-3} * a_3) + \ldots (10^{0} * a_n)$ \\

A summary of divisibility rules can be written as follows. After this table, derivation of each rule will be provided

\begin{table}[h!]
    \centering
    \begin{tabular}{|| c | c ||}
         \hline
         \textbf{Number} & \textbf{Rule} \\
         \hline
         $2^n$ or $5^n$ & Find remainder of last $n$ digits of number \\ 
         $3$ or $9$ & Sum of digits must be divisible by 3 or 9 respectively \\ 
         $11$ & Sum of digits at odd place - sum of digits at even place must be 0 or multiple of 11 \\ 
         \hline
    \end{tabular}
    \caption{Divisibility Rules}
\end{table}

For the below subsections, let us take a 6 digit number $abcdef$. We can write it as $(10^5 * a) + (10^4 * b) + (10^3 * c) + (10^2 * d) + (10^1 * e) + (10^0 * f)$. 

\subsection{Divisibility rules of $2^n$ or $5^n$}

Let us try to find the remainder of this number with $25 = 5^2$

\begin{align*}
    \remFrac{abcdef}{25} &= \remFrac{(10^5 * a) + (10^4 * b) + (10^3 * c) + (10^2 * d) + (10^1 * e) + (10^0 * f)}{25} \\
    &= \remFrac{(10^5 * a)}{25} + \remFrac{(10^4 * b)}{25} + \remFrac{(10^3 * c)}{25} + \remFrac{(10^2 * d)}{25} + \remFrac{(10^1 * e)}{25} + \remFrac{(10^0 * f)}{25} \\
    &= 0 + 0 + 0 + 0 + \remFrac{(10^1 * e)}{25} + \remFrac{(10^0 * f)}{25} \tag{All numbers $> 10^2$ contain $5^2$} \\
    &= \remFrac{10e + f}{25} \tag{Equivalent to last 2 digits of number} \\
\end{align*}

We can see that since $25 = 5^2 \implies n = 2$ and we would need to find $\remFrac{10e + f}{25}$. This is equivalent to finding remainder of last two digits with 25. Similarly, if we were dividing by $4 = 2^2$, we would get a similar result as all powers of 10 $\geq 10^2$ contain $2^2$. \\

Let us test with $8 = 2^3$ (equivalent to $125 = 5^3$)

\begin{align*}
    \remFrac{abcdef}{8} &= \remFrac{(10^5 * a) + (10^4 * b) + (10^3 * c) + (10^2 * d) + (10^1 * e) + (10^0 * f)}{8} \\
    &= \remFrac{(10^5 * a)}{8} + \remFrac{(10^4 * b)}{8} + \remFrac{(10^3 * c)}{8} + \remFrac{(10^2 * d)}{8} + \remFrac{(10^1 * e)}{8} + \remFrac{(10^0 * f)}{8} \\
    &= 0 + 0 + 0 + \remFrac{(10^2 * d)}{8} + \remFrac{(10^1 * e)}{8} + \remFrac{(10^0 * f)}{8} \tag{All numbers $> 10^3$ contain $2^3$} \\
    &= \remFrac{100d + 10e + f}{8} \tag{Equivalent to last 3 digits}
\end{align*}

\subsection{Divisibility rule of 3 or 9}

Let us try to divide by 3

\begin{align*}
    \remFrac{abcdef}{3} &= \remFrac{(10^5 * a) + (10^4 * b) + (10^3 * c) + (10^2 * d) + (10^1 * e) + (10^0 * f)}{3} \\
    &= \remFrac{(10^5 * a)}{3} + \remFrac{(10^4 * b)}{3} + \remFrac{(10^3 * c)}{3} + \remFrac{(10^2 * d)}{3} + \remFrac{(10^1 * e)}{3} + \remFrac{(10^0 * f)}{3} \\
    &= \remFrac{a + b + c + d + e + f}{3} \tag{$10^n \mod 3 = 1, n \geq 0$} 
\end{align*}

Let us try to divide by 9

\begin{align*}
    \remFrac{abcdef}{9} &= \remFrac{(10^5 * a) + (10^4 * b) + (10^3 * c) + (10^2 * d) + (10^1 * e) + (10^0 * f)}{9} \\
    &= \remFrac{(10^5 * a)}{9} + \remFrac{(10^4 * b)}{9} + \remFrac{(10^3 * c)}{9} + \remFrac{(10^2 * d)}{9} + \remFrac{(10^1 * e)}{9} + \remFrac{(10^0 * f)}{9} \\
    &= \remFrac{a + b + c + d + e + f}{9} \tag{$10^n \mod 9 = 1, n \geq 0$} 
\end{align*}

\subsection{Divisibility rule of 11}

Before we begin with this, we have a peculiar property of 11. When 11 divides $10^n$ where $n$ is even, we get remainder as 1 and when $n$ is odd, we get -1. For example, $\remFrac{100}{11} = 1 $ as $100 = 11 * 9 + 1$. However, for $\remFrac{1000}{11} = -1 = 10$ as $1000 = 11 * 90 + 10$. Thus, we can define the following

\begin{equation*}
    \remFrac{10^n}{11} = 
    \begin{cases}
        1, &\text{When $n$ is even} \\
        -1,&\text{When $n$ is odd}
    \end{cases}
\end{equation*}

We will use the above result to simplify our calculation. Finding $\remFrac{abcde}{11}$

\begin{align*}
    \remFrac{abcdef}{11} &= \remFrac{(10^5 * a) + (10^4 * b) + (10^3 * c) + (10^2 * d) + (10^1 * e) + (10^0 * f)}{11} \\
    &= \remFrac{(10^5 * a)}{11} + \remFrac{(10^4 * b)}{11} +\remFrac{(10^3 * c)}{11} + \remFrac{(10^2 * d)}{11} + \remFrac{(10^1 * e)}{11} + \remFrac{(10^0 * f)}{11} \\
    &=\remFrac{-1 * (a + c + e)}{11} + \remFrac{b + d + f}{11} \\
    &=\remFrac{(b + d + f) - (a + c + e)}{11}
\end{align*}

For $\remFrac{(b + d + f) - (a + c + e)}{11}$ to be 0, $(b + d) - (a + c + e)$ must be either 0 or a multiple of 11

\subsection{Practice Questions}

\SampleQuestion{$N$ is a number with 634 digits. It is defined as $N = 1234567 \ldots$ till the number has 634 digits. Find $\remFrac{N}{32}$}

$32 = 2^5 \implies$ we need to find remainder of the last 5 digits only. We now need to find the value of $N$

\begin{itemize}
    \item The first 9 digits of $N$ are $123456789$

    \item After this, we will get two digits per number until we reach 100 $\implies$ number of digits = $(99 - 10 + 1) * 2 = 180$ digits (There are 90 numbers between 99 and 10 inclusive. Each number has 2 digits). 

    \item Total number of digits till now = $180 + 9 = 189$

    \item Now, from 100 to 999, we will have 3 digits $\implies$ number of digits = $(999 - 100 + 1)*3 = 900*3 = 2700$. This is too much

    \item Remaining digits = $634 - 189 = 445$. The number of 3 digit numbers which will provide the remaining digits is $\dfrac{445}{3} = \dfrac{444}{3} + \dfrac{1}{3} = 148 + \dfrac{1}{3}$. This means that 148 numbers will give all digits and hundred's digit of 149th number will give the last digit.

    \item 148th number in $\inRange{100}{999} = 100 + 148 -1 = 247$. ($-1$ because 100 is included) 

    \item The last few digits of number are written as $2462472 \implies$ last 5 digits = $62472$

    \item Now, we need to check $\remFrac{62472}{32} = 8$
\end{itemize}

Therefore, remainder = 8

\section{General Methodology of deriving divisibility rules}

The point of this section is to introduce a general methodology for finding divisibility rules. They help in finding remainders of numbers for which we don't have a straightforward rule. 

\begin{NOTE}
    There is a limitation however. To derive a rule for divisibility rule, the denominator must provide a remainder of 1 or -1 when it divides any power of 10
\end{NOTE}

The general methodology is that for a denominator $d$, we need to find the value of $x$ for which $\remFrac{10^x}{d} = \pm1$. We will then split the original number in groups (\textbf{start grouping from right}) of $x$ while taking powers of 10 common in the group. Let us derive the divisibility rule for 33

\begin{NOTE}
    The first group of numbers from right will always have positive sign.
\end{NOTE}

\subsection{Rule with 33}

Let us assume that the number is $abcdef$. Since we are dividing by 33, let us find the value of $x$ for which $\remFrac{10^x}{33} = \pm1$.

\begin{align*}
    \remFrac{10^1}{33} &= 10 \\
    \remFrac{10^2}{33} &= 1 \\
\end{align*}

Therefore, $x=2$. We now need to split the number $abcdef$ in groups of 2 and take the power of 10 as common

\begin{align*}
    abcdef &= 10^5a + 10^4b + 10^3c + 10^2d + 10^1e + 10^0f \\
    &= (10^5a + 10^4b) + (10^3c + 10^2d) + (10^1e + 10^0f) \tag{Grouping from right} \\
    &= 10^4 * (10^1a + 10^0b) + 10^2 * (10^1c + 10^0d) + (10^1e + 10^0f) \tag{Take powers of 10 common in groups} \\
    &= 10^4 * (ab) + 10^2 * (cd) + (ef) \tag{$lm$ is equivalent to $10^l + m$}  
\end{align*}

Now, we need to find $\remFrac{abcdef}{33}$

\begin{align*}
    \remFrac{abcdef}{33} &= \remFrac{10^4 * (ab) + 10^2 * (cd) + (ef)}{33} \\
    &= \remFrac{10^4 * (ab)}{33} + \remFrac{10^2 * (cd)}{33} + \remFrac{(ef)}{33} \\
    &= \remFrac{ab + cd + ef}{33}
\end{align*}

We, therefore created a rule for checking divisibility rule with 33 : Split the number into groups of two from right and check whether the sum of these groups id divisible by 33 or not.

\subsection{Rule with 7}

We will assume the number to be $abcdefg$ (7 digit number). We will find the value of $x$ for which $\remFrac{10^x}{7} = \pm1$. 

\begin{align*}
    \remFrac{10^1}{7} &= 3 \\
    \remFrac{10^2}{7} &= 2 \\
    \remFrac{10^3}{7} &= 6 = -1
\end{align*}

Therefore, $x=3$. From the right, we will now split the number $abcdefg$ in groups of 3 and take power of 10 common

\begin{NOTE}
    Note that $\remFrac{10^3}{7} = -1$ and $\remFrac{10^6}{7} = 1$.
\end{NOTE}

\begin{align*}
    abcdefg &= 10^6a + 10^5b + 10^4c + 10^3d + 10^2e + 10^1f + 10^0g \\
    &= 10^6a + (10^5b + 10^4c + 10^3d) + (10^2e + 10^1f + 10^0g) \tag{Grouping from right} \\
    &= 10^6a + 10^3 * (10^2b + 10^1c + 10^0d) + 10^0 * (10^2e + 10^1f + 10^0g) \tag{Take powers of 10 common in groups} \\
    &= 10^6a + 10^3 * (bcd) + 10^0 * (efg) \tag{$lmn$ is $10^2l + 10^1m + n$} \\
\end{align*}

Now, we will find $\remFrac{abcdefg}{7}$

\begin{align*}
    \remFrac{abcdefg}{7} &= \remFrac{10^6a + 10^3 * (bcd) + 10^0 * (efg)}{7} \\
    &= \remFrac{10^6a}{7} + \remFrac{10^3 * bcd}{7} + \remFrac{10^0 efg}{7} \\
    &= \remFrac{a - bcd + efg}{7}
\end{align*}

We derived a condition where a number of form $abcdefg$ will be divisible by 7 when $\remFrac{a - bcd + efg}{7} = 0$

\begin{EXTRA-LEARNING}
    The condition will be similar for 13 as $\remFrac{10^3}{13} = -1$
\end{EXTRA-LEARNING}

\subsection{Questions}
\SampleQuestion{For a number $n = 12121212\ldots 100 digits$, Find $\remFrac{n}{33}$}

First, we can see that the number $n$ is a number where the digits $12$ are repeated until the number has 100 digits. We can see that the number of times $12$ will repeat is 50. We have earlier derived a divisibility rule of a number when it is divided by 33 $\implies$ sum of groups of 2 from right, when divided by 33, will give the remainder. 

The groups, in this specific case, will be groups of 12 as only 12 is repeating throughout the number. We will have 50 such groups

\begin{align*}
    \remFrac{n}{33} &= \remFrac{12 * 50}{33} \\
    &= \remFrac{600}{33} \\
    &= 6
\end{align*}

\SampleQuestion{For a number $n = 345345345 \ldots 200 digits$. Find $\remFrac{n}{7}$}

The number is a number which consists of 345 repeated until we have 200 digits. We can see that the number of times 345 repeats in the number is $\dfrac{200}{3} = 66$. However, the last 2 digit of the number will be $34$. Therefore, the number $n$ will be like $345345\ldots34534534$. 

We had earlier derived a divisibility rule for 7 : Split in groups of 3 from right and then find their sum. Every alternate group will have a positive sign and negative sign. In our case, starting from right, we will make groups 66 of 534, out of which 33 will be positive and 33 will be negative $\implies$ sum = 0. For the leftmost group, we will have 34 (because the first two digits will be written as $10^{199} * 3 + 10^{198} * 4$ and $\remFrac{10^198}{7} = 1$)

\begin{align*}
    \remFrac{34}{7} &= 6
\end{align*}

\section{Piggy Backing Divisibility Rules}

We can piggy back divisibility rules to get divisibility rules of factors of denominator. For example, we know that $\dfrac{1000}{7} = -1$. On a similar manner, $\dfrac{1000}{1001} = -1$. We can write as $1001 = 7 * 13 * 11$. For remainder to be -1 when divided by 1000, the factors of the denominator must give the same remainder

\begin{align*}
    \remFrac{1000}{7} &= -1 \\
    \remFrac{1000}{11} &= -1 \\
    \remFrac{1000}{13} &= -1
\end{align*}

The same divisibility rule which we have for 7 can be used on 11 and 13 as well (though since 11 divides 100 and gives 1 as remainder as well, use the simple rule) \\

On a similar note, we can use the same rule for different factors of denominator. We can see that $\dfrac{1000}{999} = 1$. We can write $999 = 37 * 27$. Therefore, we can see that 37 and 27 will have the same divisibility rule

\SampleQuestion{For a number $n = 741741\ldots 300 digits$. Find $\remFrac{n}{37}$}

First, let us find the value of $n$. We can see that the number is built up by repeating 741 until the number has 300 digits. The groups of 741 in the number are $\dfrac{300}{3} = 100$. Now, let's check divisibility rule for 37. The value of $x$ for which $10^x$ , when divided by 37 will give remainder as $\pm1$ is

\begin{align*}
    \remFrac{10^1}{37} &= 37 \\
    \remFrac{10^2}{37} &= 26 \\
    \remFrac{10^3}{37} &= 1
\end{align*}

Therefore, $x=3$. Since remainder is 1, we can just split the number in groups of 3 from right ,find sum of groups and then find remainder with 37

\begin{align*}
    \remFrac{n}{37} &= \remFrac{741 * 100}{37} \\
    &= \remFrac{74100}{37} \\
    &= \remFrac{74 + 100}{37} \tag{Split in groups of 3 from right and sum} \\
    &= \remFrac{174}{37} \\
    &= 26
\end{align*}

\SampleQuestion{For a number $n = 301302303\ldots400$, find remainder when $n$ is divided by 13}

For this, we need to know the number of digits. Number of digits can be calculated as $(400 - 301 + 1) * 3 = 300$. Now, let's see the pattern of 13 with different powers of 10

\begin{align*}
    \remFrac{10^1}{13} &= 10 \\
    \remFrac{10^2}{13} &= 9 \\
    \remFrac{10^3}{13} &= -1 \\
    \remFrac{10^4}{13} &= 3 \\
    \remFrac{10^5}{13} &= -9 \\
    \remFrac{10^6}{13} &= 1 
\end{align*}

We see that for first 3 digits from right, we will have a negative sign and then for 6 digits from right, we will have a positive sign. This is equivalent to having $\dfrac{300}{6} = 50$ groups of 6 where the difference is 1. 

\begin{align*}
    \remFrac{301302303\ldots400}{6} &= \remFrac{1 * 50}{13} \\
    &= \remFrac{50}{13} \\
    &= 11
\end{align*}

\SampleQuestion{For a number $n = 5555\ldots$ 602 digits, find remainder when $n$ is divided by 21}

Since denominator is not prime, we can use Chinese remainder theorem to split the denominator and find remainder with each factor $\implies \remFrac{n}{3}$ and $\remFrac{n}{7}$. 

Finding $\remFrac{n}{3}$

\begin{align*}
    \remFrac{n}{3} &= \remFrac{5 * 602}{3} \tag{For 3, remainder is sum of each digit} \\
    &= \remFrac{5}{3} * \remFrac{602}{3} \\
    &= \remFrac{2 * 2}{3} \\
    &= 1
\end{align*}

Finding remainder with 7. For 7, we know that $\remFrac{10^3}{7} = -1$ and $\remFrac{10^6}{7} = 1$. We can say that from right, every 3 digits will have alternating signs, starting from positive. In $n$, where we have 602 digits, we will have $\dfrac{602}{3} = 200$ groups of 6 where difference will be 0

\begin{align*}
    \remFrac{n}{7} &= \remFrac{55 + 0}{7} \\
    &= 6
\end{align*}

Now, we have a number $n$ which, when divided by 3 will give 1 as a remainder AND when divided by 7, will give 6 as a remainder. We need to find numbers which satisfy this property as well. (Chinese remainder theorem)

\begin{align*}
    3x + 1 &= 7y + 6 \\
    x &= \dfrac{7y + 5}{3}
    x &= 4 \tag{y = 1}
\end{align*}

Using $x=4$, we can find $3x + 1 = 13$. When 13 is divided by 21, the remainder will be 13.

\SampleQuestion{For a number $n=56475647\ldots$ 300 digits, find remainder when $n$ is divided by 101}

In $n$, we will have $\dfrac{300}{4} = 75$ groups of 5647. Now, let us discover the divisibility rule of 101

\begin{align*}
    \remFrac{10^1}{101} &= 10 \\
    \remFrac{10^2}{101} &= -1 \\
    \remFrac{10^3}{101} &= -10 = 91 \\
    \remFrac{10^4}{101} &= 1 \\
\end{align*}

For convenience, we can use the property that $\remFrac{10^4}{101} = 1$. Using this, we can make 75 groups of 4 digits and then find remainder

\begin{align*}
    \remFrac{n}{101} &= \remFrac{5647 * 75}{101} \\
    &= \remFrac{5647}{101} * \remFrac{75}{101} \\
    &= 75 * \remFrac{5647}{101} \\
    &= \remFrac{75 * 92}{101} \\
    &= \remFrac{6900}{101} \\
    &= \remFrac{-69}{101} \tag{Used $100 \mod 101 = 1$} \\
    &= 32
\end{align*}
