\section{Basics and Interesting cases}

Average is defined as $\dfrac{\text{Sum}}{\text{Number of quantities}}$. For example, the average of numbers 17,12,10 and 20 is $\dfrac{17 + 12 + 10 + 20}{4} = \dfrac{59}{4} = 14.75$. There are some special cases

When numbers are in AP (Arithmetic Progression), the average of that sequence is the middle term of the sequence

\begin{itemize}
    \item Middle term of AP is given as $\dfrac{\text{First term} + \text{Last term}}{2}$
    \begin{itemize}
        \item This middle term may or may not exist in the sequence itself : If a sequence has odd number of terms, then the average exists in the sequence but if there are even number of terms, average will not exist in sequence
        \item 2,4,6,8,10 . Average = $\dfrac{2 + 10}{2} = 6$. 6 exists in AP. 
        \item 1,3,5,7. Average = $\dfrac{1 + 7}{2} = 4$. 4 does not exist in AP
    \end{itemize}

    \item If we want to find the index of the middle term/s that resulted in average, they are as follows
    \begin{itemize}
        \item Odd number of terms : $\floor{ \dfrac{\text{Number of terms}}{2} } + 1$
        \item Even number of terms : $ \dfrac{\text{Number of terms}}{2} $, $ \dfrac{\text{Number of terms}}{2} + 1$. The average of AP is the average of these middle terms 
    \end{itemize}
\end{itemize}

\SampleQuestion{Average of 7 consecutive even integers is 36. Product of 2nd and 5th term is?}

Since we have odd number of terms in the AP, the middle term of the AP will be the average. If we write our AP as $a , a+2 , a+4 , a+6 , a+8 , a+10 , a+12 $. The middle term is $a + 6 = 36 \implies a = 30$. \\

Product of 2nd and 5th term = $(a + 2) * (a + 8) = 32 * 38 = 1216$

\SampleQuestion{Average of 12 consecutive odd integers is 30. Find the sequence}. 

The sequence is an AP with 12 terms and common difference = 2. Since we have even terms, the average is calculated as average of middle terms. Let the first term of the AP be $a$.

\begin{align*}
    30 &= \dfrac{a + (4 * 2) + a + (5 * 2)}{2} \\
    30 &= \dfrac{2a + 18}{2} \\
    &= a + 9 \\
    a &= 21
\end{align*}

Series is therefore, $21,23,25,27,29,31,33,35,37,39,41,43$

\textbf{Another approach}

We can see that the average of the sequence is 30. In a sequence of 12 terms, average is derived by average of 2 middle terms. In this question, the terms would be the 5th and 6th term. 

\begin{itemize}
    \item If the average of 2 terms is an integer $x$, then the terms must be $x-1$ and $x+1$. $\dfrac{(x-1) + (x+1)}{2} = \dfrac{2x}{2} = x$. 

    \item If the average of 2 terms is a decimal $y$, then the terms must be $y-0.5$ and $y+0.5$. For example, if average is 14.5, then the terms are 14 and 15 $\dfrac{14 + 15}{2} = \dfrac{29}{2} = 14.5$. 
\end{itemize}

Using the above, we can find the 5th and 6th terms : 30-1 and 30+1 respectively. Using that, with common difference of 2, we can find the terms

\SampleQuestion{ Average of 143 consecutive odd integers is 'P'. Average of last 67 terms is 'n'. Find 'P' in terms of 'n' }

\begin{itemize}
    \item Since average of 143 terms is $P$, we can say that $\floor{\dfrac{143}{2}} + 1 = 72^{nd}$ term is equal to $P$. 
    
    \item The last 67 terms are in the range 77th term to 143rd term ($143 - 67 + 1 = 76$). The middle term of this sequence will be $\dfrac{77 + 143}{2} = 110$. 
    
    \item $110^{th}$ term = $n$, $72^{nd}$ term = $P$. Difference = $110 - 72 = 38$. Therefore, $n = P + 76$
\end{itemize}

\begin{NOTE}
    I will be using the notation $t_n$ from now on to describe the $n^{th}$ term
\end{NOTE}

\SampleQuestion{Find the average of 1 + 3 + 5 + 7 + $\ldots$ 167}

The above is an AP with common difference of 2. We can find the average as $\dfrac{1 + 167}{2} = 84$

\newpage

\SampleQuestion{Which digit is missing in the average of numbers 9,99,999,9999 $\ldots$ 999999999 ?}

There are two ways to solve this question : Through pattern and actual calculation. I will show both

\textbf{Pattern Matching}

We are asked to find the missing digit in $\dfrac{9 + 99 + 999 + \ldots 999999999}{9}$. We can simplify by taking 9 common $\implies 1 + 11 + 111 + \ldots 111111111$

Now, we can start to notice a pattern
\begin{itemize}
    \item 1 + 11 = 12
    \item 1 + 11 + 111 = 123
    \item 1 + 11 + 111 + 1111 = 1234
    \item We can see that as we keep adding the next series of 1s, we are getting each digit. Like, when we found 1 digit number + 2 digit number + 3 digit number + 4 digit number, we got all the 4 digits
    \item We can extend this and be sure that when we add 9 numbers like this, we will get the sum 123456789
\end{itemize}

We can now say that the digit 0 is missing    

\textbf{Actually Finding the Sum}
We can find the sum of expression $9 + 99 + 999 + \ldots 999999999$ by using geometric progression formula

\begin{align*}
    9 + 99 + 999 + \ldots 999999999 &= (10 - 1) + (10^2 - 1) + (10^3 - 1) + \ldots (10^9 - 1) \\
    &= (10 + 10^2 + 10^3 + \ldots 10^9 ) - 9 \tag{There are nine "1"} \\
    &= \dfrac{10 * (10^9 - 1)}{10 - 1} - 9 \tag{GP sum formula with $r = 10$ and $n = 9$} \\
    &= 1111111110 - 9 \\
    &= 1111111101
\end{align*}

Average is $\dfrac{1111111101}{9} = 123456789 \implies $ Missing digit = 0

\SampleQuestion{The sales of a company in January 2012 was Rs 348 crores. In Feb and March, sales were Rs 364 crores and Rs 380 crores respectively. If sales increase in a similar trend till December of that year, find average sales in 2012}

We can see that 348, 364 and 380 are in an AP with $a = 348, d = 16 \text{and} n = 12$. Therefore, we can find average by taking sum of sale of january and december and dividing by 2

\begin{itemize}
    \item Sales in December = $348 + (11*16) = 348 + 176 = 524$
    \item Average = $\dfrac{348 + 524}{2} = 436$
\end{itemize}

Answer = Rs 436 crores

\newpage











\section{Average Increase and Decrease}

This is a class of questions where we are given an average, the change in average and number of quantities. Refer to the following examples 

\SampleQuestion{The average weight of a group of 15 friends increases by 1 Kg when a person joins the group. Find the weight of the person who joined the group if the initial average weight of the group is 48Kg}

\begin{multicols}{2}

    \textbf{Mathematics Method}

    Let $S$ be the sum of weights of 15 friends. According to the question 
    \begin{align*}
        \dfrac{S}{15} &= 48 \\
        S &= 48 * 15 \\
        &= 720
    \end{align*}
    
    By adding a new person, average becomes 49. Let the weight of this new person be $x$
    
    \begin{align*}
        \dfrac{S + x}{16} &= 49 \\
        720 + x &= 49 * 16 \\
        x &= 784 - 720 \\
        x &= 64
    \end{align*}
    

    \columnbreak

    \textbf{Logic Method}

    \begin{itemize}
        \item Let us assume that on an average, each friend in the friend group weighs 48kg. We can make this assumption as the average will still be 48 if each friend weighs 48 Kg
        \item Now, a new friend joined in and because of this, the average weight increased by 1
        \item This is the equivalent of each friend being 49kg on an average. For each friend, the average weight increased by 1
        \begin{itemize}
            \item The average would not have changed if the new friend was 48Kg
            \item Since the average increased, friend must be heavier than 48Kg
            \item Each friend's average weight increased by 1 Kg therefore, the new friend must weight $48 + (1 * 16) = 64$ as there are 16 friends now
        \end{itemize}
    \end{itemize}

\end{multicols}

\SampleQuestion{The average weight of a group of 20 friends increases by 2 Kg when a person joins the group. Find the weight of the person who joined the group if the initial average weight of the group is 60Kg}

\begin{itemize}
    \item 20 friends where each friend on an average weighed 60Kg
    \item New friend (friend 21) joined which led to increase in average weight of each friend by 2
    \item The average weight of 21 friend increased by 2 therefore weight of friend 21 = 60 + (21 * 2) = 102
\end{itemize}



\href{https://youtu.be/q-ZUkah-xys?feature=shared&t=1381}{Continue from here}