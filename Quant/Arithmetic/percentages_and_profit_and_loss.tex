\section{Important Fraction to Percentage (and vice-versa)}

\begin{multicols}{3}

\textbf{Primes}

\begin{itemize}
    \item $\dfrac{100}{2} = 50\%$
    \item $\dfrac{100}{3} = 33.33\%$
    \item $\dfrac{100}{5} = 20\%$
    \item $\dfrac{100}{7} = 14.28\%$
    \item $\dfrac{100}{11} = 9.09\%$
    \item $\dfrac{100}{13} = 7.69\%$
    \item $\dfrac{100}{17} = 5.88\%$
    \item $\dfrac{100}{19} = 5.26\%$
    \item $\dfrac{100}{21} = 4.76\%$
    \item $\dfrac{100}{23} = 4.34\%$
\end{itemize}

\columnbreak

\textbf{Composites}

\begin{itemize}
    \item $\dfrac{100}{4} = 25\%$ \hspace{0.2cm}($\dfrac{100}{2} * \dfrac{1}{2})$
    \item $\dfrac{100}{6} = 16.66\%$ \hspace{0.2cm}($\dfrac{100}{3} * \dfrac{1}{2})$
    \item $\dfrac{100}{8} = 12.5\%$ \hspace{0.2cm}($\dfrac{100}{4} * \dfrac{1}{2})$
    \item $\dfrac{100}{9} = 11.11\%$ \hspace{0.2cm} ($\dfrac{100}{3} * \dfrac{1}{3})$
    \item $\dfrac{100}{10} = 10\%$ 
    \item $\dfrac{100}{12} = 8.33\%$ \hspace{0.2cm} ($\dfrac{100}{6} * \dfrac{1}{2})$
    \item $\dfrac{100}{14} = 7.14\%$ \hspace{0.2cm} ($\dfrac{100}{7} * \dfrac{1}{2})$
    \item $\dfrac{100}{15} = 6.33\%$ \hspace{0.2cm} ($\dfrac{100}{5} * \dfrac{1}{3})$
    \item $\dfrac{100}{16} = 6.25\%$ \hspace{0.2cm} ($\dfrac{100}{8} * \dfrac{1}{2})$
    \item $\dfrac{100}{18} = 5.55\%$ \hspace{0.2cm} ($\dfrac{100}{9} * \dfrac{1}{2})$
    \item $\dfrac{100}{20} = 5\%$ 
    \item $\dfrac{100}{22} = 4.545\%$ \hspace{0.2cm} ($\dfrac{100}{11} * \dfrac{1}{2})$
    \item $\dfrac{100}{24} = 4.166\%$ \hspace{0.2cm} ($\dfrac{100}{12} * \dfrac{1}{2})$
    \item $\dfrac{100}{25} = 4\%$
\end{itemize}

\columnbreak

\textbf{Special Composites}

\begin{itemize}
    \item $37.5\% = \dfrac{3}{8}$ \hspace{0.2cm}($\dfrac{100}{8} * 3$)
    \item $62.5\% = \dfrac{5}{8}$ \hspace{0.2cm}($\dfrac{100}{8} * 5$)
    \item $87.5\% = \dfrac{7}{8}$ \hspace{0.2cm}($\dfrac{100}{8} * 7$)
    \item $18.75\% = \dfrac{3}{16}$ \hspace{0.2cm}($\dfrac{100}{16} * 3$)
    \item $83.33\% = \dfrac{5}{6}$ \hspace{0.2cm}($\dfrac{100}{6} * 5$)
    \item $28.56\% = \dfrac{2}{7}$ \hspace{0.2cm}($\dfrac{1}{7} * 2$)
    \item $43.75\% = \dfrac{7}{16}$ \hspace{0.2cm}($50\% - 6.25\%$)
    \item $56.25\% = \dfrac{9}{16}$ ($50\% + 6.25\%$)
\end{itemize}

\end{multicols}
\newpage












\section{Multiplying Factor}
If a number is increased or decreased by a certain percentage, there exists a fraction which when multiplied with the number, will give the new value. This fraction is called multiplying factor. For example, if 100 is increased by 25\%, we write
\begin{align*}
    \text{New number} &= 100 + \dfrac{25}{100} * 100 \\
    &= 125 \\
    &= 1.25 * 100 \tag{1.25 is the multiplication factor}
\end{align*}

It is better to deal with this in fraction. In above case, instead of using 25\%, we will use its fraction form : $\dfrac{1}{4}$

\begin{align*}
    \text{Multiplying factor} &= 1 + \dfrac{1}{4} \\
    &= \dfrac{5}{4} = 1.25
\end{align*}

\textbf{Finding multiplication factor}
We convert the percentage to its fraction form and then apply the formula below

\begin{equation*}
    \text{Multiplication factor} = \begin{cases}
        1 + f & , \text{When number is increased by fraction $f$} \\
        1 - f & , \text{When number is decreased by fraction $f$} \\
    \end{cases}    
\end{equation*}

\SampleQuestion{648 is decreased by 37.5\%. Find the new value}
\begin{itemize}
    \item Fraction = $\dfrac{3}{8}$
    \item Multiplying factor = $1 - \dfrac{3}{8} = \dfrac{5}{8}$
    \item New Number = $\dfrac{5}{8} * 648 = 405$
\end{itemize}

\SampleQuestion{1212 is increased by 83.33\%. Find the new value}
\begin{itemize}
    \item Fraction = $\dfrac{5}{6}$
    \item Multiplying factor = $1 + \dfrac{5}{6} = \dfrac{11}{6}$
    \item New Number = $\dfrac{11}{6} * 1212 = 202 * 11 = 2222$
\end{itemize}

\newpage


\section{Successive increases and decreases}
If a number is increased or decreased by a bunch of percentages, it will be calculated differently. For example, to calculate the percentage change when a number is first increased by 25\% , then increased by 5\% and then increased by 10\% . For the sake of easy calculation, let us assume that the number is initially 100

\begin{itemize}
    \item Increase by 25\% $\implies 100 + \dfrac{25}{100} * 100 = 125$ 
    \item Increase by 10\% $\implies 125 + \dfrac{10}{100} * 125 = 137.5$ 
    \item Increase by 5\% $\implies 137.5 + \dfrac{5}{100} * 137.5 = 137.5 + 6.875 = 144.375$ 
\end{itemize}

\begin{NOTE}
    It is recommended to start from the "larger" percentages while finding increases / decreases. In MCQ type questions, it will allow us to reach closer to the actual value quickly
\end{NOTE}

When the percentages are too complex (like, contains decimals), try converting them to fraction and then find multiplication factor of each

\SampleQuestion{A number is increased by 18.75\% and decreased by 12.5\%. Find the percentage change in number}

Since the fractions are "complicated" (contain fractions), we can find multiplying factor for each case

\begin{multicols}{2}

    Increase by 18.75\% : $1 + \dfrac{3}{16} = \dfrac{19}{16}$. 
    
    New number = $\dfrac{19}{16}$
    
    \columnbreak
    Decrease by 12.5\% : $1 - \dfrac{1}{8} = \dfrac{7}{8}$. 
    
    New number is $\dfrac{19}{16} * \dfrac{7}{8} = \dfrac{133}{128}$
    
\end{multicols}
\vspace{0.2cm}

\begin{align*}
    \text{Percentage change} &= \dfrac{\text{New number} - \text{Original number}}{\text{Original number}} \\
    &= \dfrac{\dfrac{133}{128} - 1}{1} \\
    &= \dfrac{5}{128} = 3.9\% \\
\end{align*}


\newpage

\SampleQuestion{A number is increased by 166.66\% and then decreased by 37.5\% and 18.75\%. Find percentage change}

First, let's find multiplication factors
\begin{multicols}{3}
    166.66\% = 100\% + 66.66\%
    
    \begin{itemize}
        \item Fraction = $1 + \dfrac{2}{3} = \dfrac{5}{3}$
        \item Multiplication Factor = $1 + \dfrac{5}{3} = \dfrac{8}{3}$
    \end{itemize}
    \columnbreak

    37.5\%
    \begin{itemize}
        \item Fraction = $\dfrac{3}{8}$
        \item Multiplication Factor = $1 - \dfrac{3}{8} = \dfrac{5}{8}$
    \end{itemize}
    \columnbreak

    18.75\%
    \begin{itemize}
        \item Fraction = $\dfrac{3}{16}$
        \item Multiplication Factor = $1 - \dfrac{3}{16} = \dfrac{13}{16}$
    \end{itemize}
    
\end{multicols}
\vspace{0.2cm}

New number = $\dfrac{8}{3} * \dfrac{5}{8} * \dfrac{13}{16} = \dfrac{65}{48}$
\vspace{0.2cm}

Percentage change = $1 - \dfrac{65}{48} = \dfrac{17}{48} = \dfrac{16}{48} + \dfrac{1}{48} = 0.333 + 0.02 = 0.353 = 35.3\%$
\newpage





















\section{Percentage Change when a number increases / decreases}
We have an interesting property regarding the percentage change between two numbers. The cases are defined as follows

\begin{itemize}
    \item $A \xLeftrightarrow[\text{Decrease by $\dfrac{n}{n+d}$}]{\text{Increase by $\dfrac{n}{d}$}} B$

    \item $B \xLeftrightarrow[\text{Increase by $\dfrac{n}{d}$}]{\text{Decrease by $\dfrac{n}{n+d}$}} A$ 

\end{itemize}

To demonstrate the above results, see these questions

\SampleQuestion{We are given two numbers 70 and 80. Find the following}
\begin{enumerate}
    \item Percentage by which 70 should be increased such that we get 80 as a result
    \item Percentage by which 80 should be decreased such that we get 70 as a result
\end{enumerate}

The general way to solve this question is as follows
\begin{itemize}
    \item $70 \xrightarrow{} 80$ : Percentage difference = $\dfrac{80 - 70}{70} * 100 = \dfrac{1}{7} = 14.28\%$

    \item $80 \xrightarrow{} 70$ : Percentage difference = $\dfrac{70 - 80}{80} * 100 = \dfrac{-1}{8} = -12.5\%$
\end{itemize}

We can use the above result to simplify the question : We have discovered that the percentage that increases 70 to 80 is 14.28\% or $\dfrac{1}{7}$. In our solution above, $\dfrac{n}{d} = \dfrac{1}{7} \implies n = 1, d = 7$

$70 \xLeftrightarrow[\text{Decrease by $\dfrac{1}{1+7}$}]{\text{Increase by $\dfrac{1}{7}$}} 80 \implies 70 \xLeftrightarrow[\text{Decrease by $\dfrac{1}{8} = 12.5\%$}]{\text{Increase by $\dfrac{1}{7} = 14.28\%$}} 80 $

\SampleQuestion{130 is increased by a percentage $p$ to 160. Find percentage by which 160 should be decreased such that we get 130}

Percentage difference from $130 \xrightarrow{} 160 = \dfrac{160 - 130}{130} = \dfrac{3}{13}$. Now, applying our result

$130 \xLeftrightarrow[\text{Decrease by $\dfrac{3}{13 + 3}$}]{\text{Increase by $\dfrac{3}{13}$}} 160 \implies 130 \xLeftrightarrow[\text{Decrease by $\dfrac{3}{16} = 18.75\% $}]{\text{Increase by $\dfrac{3}{13} = 23.1\%$}} 160$

\SampleQuestion{A is 40\% more than B. By \% is B less than A}
We can write 40\%  as $\dfrac{40}{100} = \dfrac{2}{5}$

\begin{itemize}
    \item $A \xrightarrow{\text{Increase by $\dfrac{2}{5}$}} B$
    \item $A \xLeftrightarrow[\text{Decrease by $\dfrac{2}{2+5} = \dfrac{2}{7} = 28.56\% $}]{\text{Increase by $\dfrac{2}{5}$}} B$
\end{itemize}

\SampleQuestion{A's salary is 40\% less than B. Find how much salary of B is greater than A}

We can write 40\% as $\dfrac{40}{100} = \dfrac{2}{5}$

\begin{itemize}
    \item $B \xleftarrow[\text{Decrease by $\dfrac{2}{5}$ } ]{} A$
    \item $B \xLeftrightarrow[\text{Decrease by $\dfrac{2}{5}$ } ]{\text{Increase by $\dfrac{2}{5 - 2} = \dfrac{2}{3} = 66.66\%$}} A $
\end{itemize}

\subsection{Expenditure, Price and Consumption}
Expenditure is defined as product of price and the quantity of item consumed. Mathematically, $E = P * C$ where E = expenditure, P = price and C = quantity of item consumed. We can see that if expenditure is constant, $P = \dfrac{E}{C} \implies $ Price is inversely proportional to quantity of item consumed.

\begin{itemize}
    \item If price increases, to keep expenditure same, less quantity must be consumed that is, if P increases by $\dfrac{n}{d}$, then C will decrease by $\dfrac{n}{n + d}$
    
    \item If price decreases, to keep expenditure same, more quantity must be consumed that is, if P decreases by $\dfrac{n}{n + d}$, then C will decrease by $\dfrac{n}{d}$
\end{itemize}

\begin{NOTE}
    $\dfrac{n}{d}$ in the above expressions mean $\dfrac{n}{d} * 100 \%$
\end{NOTE}

\SampleQuestion{If price of sugar increases by 30\%, then to keep expenditure constant, how much consumption should be reduced?}

Price increases by $\dfrac{30}{100} = \dfrac{3}{10}$. Here, $n = 3. d = 10$. 

Therefore, consumption will decrease by $\dfrac{n}{n + d} = \dfrac{3}{10 + 3} = \dfrac{3}{13} = 23.1\% $

\newpage

\section{Profit and Loss}
We have three terms. Usually, we "look" at the situation from a shopkeeper's perspective that is, a shopkeeper selling a product to a customer

\begin{itemize}
    \item Cost Price (CP) : The price at which the shopkeeper buys an item
    \item Mark Price (MP) : The price at which the shopkeeper wants to sell an item
    \item Selling Price (SP) : The price at which the shopkeeper actually sells the product to customer. This could be same as MP or some discount could be offered on MP.
\end{itemize}

For exmaple, let us take the following situation
\begin{itemize}
    \item Shopkeeper buys an item for 500. Here, CP = 500
    \item He marks the price upto 1000 $\implies$ MP = 1000
    \item He sells the item to customer after offering 20\% discount on MP $\implies SP = \dfrac{80}{100} * 1000 = 800$
\end{itemize}

Now, we calculate a few things
\begin{itemize}
    \item Profit : When we sell a product at a price higher than we bought it, we get a profit. $P\% = \dfrac{P}{CP} * 100$
    \item Loss : When we sell a product at a price lower than we bought it, we get a loss. $L\% = \dfrac{L}{CP} * 100$
    \item \textbf{Profit and Loss are calculated on basis of Cost Price}
    \item Discount : When we sell a product for less than what we wanted to sell (MP), we offer a discount to customer. \textbf{Discount is calculated on basis of Mark  Price}. $D\% = \dfrac{D}{MP} * 100$
\end{itemize}

An illustration to remember the above relations
$$
\text{CP} \xrightarrow{\text{Profit or loss}} \text{SP} \xleftarrow{\text{Discount}} \text{MP}
$$

\SampleQuestion{If we offer 20\% discount, we get 25\% profit. Find out the profit \% when we offer 10\% discount}

Let CP = $100x$. Using above info, we can find the selling price :  SP = $(1 + \dfrac{1}{4}) * 100x = 125x$ as we have a profit of 25\%. Let us depict the relationship as 

$$
\text{CP}(100x) \xrightarrow{\text{Profit of 25\%}} \text{SP}(125x) \xleftarrow{\text{Discount of 20\%}} \text{MP}
$$

If we focus on the link between SP and MP, we can find MP by using the concept of percentage change 
\begin{itemize}
    \item Discount (decrease) of 20\% = $\dfrac{20}{100} = \dfrac{1}{5}$, ($n = 1,n + d = 5 \implies d = 4$)
    
    \item We will thus, have an increase of $\dfrac{n}{d} = \dfrac{1}{4} = 25\% $

    \item Thus, MP = $(1 + \dfrac{1}{4}) * 125x = 125x + 31.25x = 156.25x$
\end{itemize}    

Now, if discount is 10\%, the new selling price will be 10\% decrease in $156.25x = 156.25 - 15.625 = (156 - 15) + (0.250 - 0.625) = 141 - 0.375 = 140.625$. \\

Profit \% = $\dfrac{140.625x - 100x}{100x} * 100 = 40.6\%$

\SampleQuestion{On 25\% discount, we get 20\% profit. On 5\% discount, how much profit \% will we get?}

\begin{itemize}
    \item Let CP = $100x$
    \item Profit is 20\% therefore SP = $120x$.
    \item MP and SP are related as follows : When MP is reduced by 25\%, we get SP.
    \begin{itemize}
        \item We can write 25\% as $\dfrac{1}{4} \implies \dfrac{n}{n+d} = \dfrac{1}{4}$. 
        \item Therefore, when SP is increased by $\dfrac{n}{d} = \dfrac{1}{3} = 33.33\%$ , we will get MP
        \item MP = $( 1 + \dfrac{1}{3} ) * 120x = 160x$
    \end{itemize}
\end{itemize}

Now, if we offer 5\% discount, the selling price will be $(1 - \dfrac{1}{20}) * 160x = 152x$. 

Profit = $152x - 100x = 52x \implies 52\% $ profit 


\SampleQuestion{If profit is 50\% and discount is 20\%, find the markup percentage}

Let CP = $100x$

\begin{itemize}
    \item Since profit is 50\%, SP = $150x$
    \item We have $SP \xleftarrow{\text{Discount 20\%}} MP$. 
    \begin{itemize}
        \item 20\% = $\dfrac{20}{100} = \dfrac{1}{5} \implies n = 1, n+d = 5$
        \item MP will therefore, be $\dfrac{n}{d} = \frac{1}{4} = 25\%$ more than SP
        \item MP = $150x + $ 25\% of $150x$ = $150x + 37.5x = 187.5x$
    \end{itemize}
\end{itemize}

Markup percentage = $\dfrac{MP - CP}{CP} = 87.5\%$

\SampleQuestion{If profit = 75\% after giving discount of 16.66\%, find the following}
\begin{enumerate}
    \item Markup \%
    \item Profit \% if discount offered on markup is now 28.56\% instead of 16.66\%
    \item Discount \% if profit \% is 36.5\%
\end{enumerate}

\vspace{0.5cm}
Let CP = $100x$
\begin{itemize}
    \item Since profit \% = 75, SP = $175x$
    \item Discount is $16.66 = \dfrac{1}{6}, n = 1, n+d = 16$
    \item MP will be $\dfrac{n}{d} = \dfrac{1}{5}$ more than SP
    \item $MP = \dfrac{6}{5} * 175x = 210x$
    \item Mark \% = 110\%
\end{itemize}

\textbf{Discount is 28.56\%}
\begin{itemize}
    \item 28.56\% = $\dfrac{2}{7}$
    \item SP = $\dfrac{5}{7} * 210 = 150x$
    \item P\% = $\dfrac{SP - CP}{CP} * 100 = 50\%$
\end{itemize}

\textbf{Discount \% when profit \% is 36.5}
\begin{itemize}
    \item SP = $136.5x$
    \item MP = $210x$
    \item Discount \% = $\dfrac{136.5x - 210x}{210x} * 100 = \dfrac{73.5}{210} * 100 = 35\%$
\end{itemize}

\SampleQuestion{The marked price of a table is Rs 4800. What will be the selling price if two successive discounts of 10\% and 37.5\% are allowed on it?}

We have two discounts that are applied on the mark price. We can write the discounts in fraction form for easy calculation : 10\% = $\dfrac{1}{10}$ and 37.5\% = $\dfrac{3}{8}$. For the sake of verbosity, let us calculate price after each discount (note that we can also do it in one expression as well)

\begin{itemize}
    \item 10\% discount
    \begin{itemize}
        \item Discount = $\dfrac{1}{10}$
        \item Remaining value = $1 - \dfrac{1}{10} = \dfrac{9}{10} $ of original value
        \item New value = $\dfrac{9}{10} * 4800 = 4320$
    \end{itemize}

    \item 37.5\% discount
    \begin{itemize}
        \item Discount = $\dfrac{3}{8}$
        \item Remaining value = $1 - \dfrac{3}{8} = \dfrac{5}{8}$ of original value
        \item New value = $\dfrac{5}{8} * 4320 = 5 * 540 = 2700$
    \end{itemize}
\end{itemize}

Selling price = 2700

\SampleQuestion{A shopkeeper marked his goods at 166.66\% above the cost price and gave 2 successive discounts of 37.5\% and 18.75\%. Find his profit/loss \%}

We can calculate the percentage values above in fractions for ease of calculation
\begin{itemize}
    \item 166.66\% = 1 + $\dfrac{2}{3}$ = $\dfrac{5}{3}$
    \item 37.5\% = $\dfrac{3}{8}$
    \item 18.75\% = $\dfrac{3}{16}$
\end{itemize}

\begin{NOTE}
    When it is easy to convert discount values to fractions for calculation, prefer that. In those cases, instead of taking CP = 100, we can take CP = 1
\end{NOTE}

Let CP = 1. The marked price MP will be calculated as $CP * (1 + \dfrac{5}{3}) = \dfrac{8}{3}$. On this marked price, we will have the following values

\begin{itemize}
    \item Discount of 37.5\% : New value = $\text{oldValue} * ( 1 - \dfrac{3}{8}) = \text{oldValue} * \dfrac{5}{8}$. Let us call this \textbf{discount1Value}

    \item Discount of 18.75\% : New value = $\text{discount1Value} * ( 1 - \dfrac{3}{16}) = \text{discount1Value} * \dfrac{13}{16}$
\end{itemize}

We can calculate final value as $1 * \dfrac{5}{8} * \dfrac{13}{16} = \dfrac{65}{48}$

P\% = $\dfrac{\dfrac{65}{48} - 1}{1} * 100 = \dfrac{17}{48} * 1 = 35.33\%$

\SampleQuestion{After allowing a discount of 11.11\%, a trader makes a profit of 14.28\% . What is the markup percentage?}

We can write the percentages as fractions : 11.11\% = $\dfrac{1}{9}$ and 14.28\% = $\dfrac{1}{7}$

\begin{itemize}
    \item SP = $(1 + \dfrac{1}{7}) * CP \implies SP = \dfrac{8}{7} CP$
    \item However, $SP = (1 - \dfrac{1}{9}) MP \implies SP = \dfrac{8}{9} MP$
\end{itemize}

We can use the above to form an equation : 
\begin{align*}
    SP &= \dfrac{8}{9} MP \\
    \dfrac{8}{7} CP &= \dfrac{8}{9} MP \\
    MP &= \dfrac{9}{7} CP
\end{align*}

Markup = $CP - \dfrac{9}{7} CP = \dfrac{2}{7}CP$. Therefore, markup percentage = 28.56\%

\SampleQuestion{A person sold an article at a profit of 15\%. Had he bought it for 15\% less price, and sold it for Rs 150 more, he would have gained 50\%. Find the cost price of article}

Let CP = $100x$. Converting the above question in mathematical statements
\begin{itemize}
    \item SP = $100x$ + 15\% of $100x$ as profit is 15\% $\implies SP = 115x$
    
    \item If he had brought it for 15\% less price $\implies CP_2 = 85x$ and sold it for Rs150 more $\implies SP = 115x + 150$, he would have gained profit of 50\%
    \begin{itemize}
        \item Since profit is 50\% when the person bought the product for $CP_2$, we can find the selling price $\implies$ SP = $85x$ + 50\% of $85x \implies SP = \dfrac{255}{2}x$
        \item Above, we determined that $SP = 115x + 150$
    \end{itemize}
\end{itemize}

Comparing the selling price
\begin{align*}
    \dfrac{255}{2}x &= 115x + 150 \\
    255x &= 230x + 300 \\
    25x &= 300 \\
    x &= 12
\end{align*}

Cost price = $100 * 12 = 1200$

\SampleQuestion{On selling 17 balls at Rs 720, there is a loss equal to the cost price of 5 balls. The cost of the ball is?}

We sold 17 balls at a selling price of 720 and incurred loss of an amount equal to price of 5 balls. We can put this into an equation
\begin{align*}
    17CP - SP &= 5CP \tag{Cost price of 17 balls = 17CP, selling price = SP} \\
    12CP &= SP \\
    CP &= \dfrac{720}{12} \\
    CP &= 60
\end{align*}


\SampleQuestion{'A' purchases a book at a discount of 24\% on the listed price. He decided to sell it to B to avoid loss. If he wants to make a profit of 25\% after allowing a discount of 20\%, by what percent should his marked price be greater than original price?}

\begin{itemize}
    \item Let the listed price of book be $100x$. 'A' purchased a book at a discount of 24\% $\implies 76x$. 
    \item Now, after purchasing the book for $76x$, he wants to sell it to get a profit of 25\% $\implies SP = (1 + \dfrac{1}{4}) * 76x = 95x$
    \item Since he also wants to give a discount of 20\%, his selling price should be 80\% of markup price $\implies SP = \dfrac{8}{10} MP \implies MP = \dfrac{95x * 10}{8} = \dfrac{952}{8} - \dfrac{2}{8} = 119 - 0.25 = 118.75$
\end{itemize}

Markup percentage = $\dfrac{118.75x - 100x}{100x} * 100 = 18.75\%$

\begin{EXTRA-LEARNING}
    We can also find the markup value by using the inverse relation between markup and selling price. If we are reducing selling price by 20\% = $\dfrac{1}{5} = \dfrac{n}{n+d}$, then if we increase the selling price by $\dfrac{n}{d} = \dfrac{1}{4} = 25\%$, we should get the markup percentage. 

    Since we already have a value of SP, we can use it to find MP (instead of doing the math we did)
\end{EXTRA-LEARNING}
\newpage

















\subsection{Selling and Cost Price based on quantity of items}
This is a category of questions where compare cost and selling price in terms of quantities. Refer to the below question and see the three approaches we can take

\SampleQuestion{Selling price of 25 articles = Cost price of 30 articles. Find the profit\% }

\textbf{Approach 1 : Use variables}

We can use variables like CP and SP and calculate values
\begin{align*}
    25 SP &= 30 CP \\
    SP &= \dfrac{30}{25} CP \\
    SP &= \dfrac{6}{5} CP
\end{align*}

Profit\% = $\dfrac{\dfrac{6}{5} CP - CP}{CP} * 100 = \dfrac{1}{5} * 100 = 20\%$

\textbf{Approach 2 : Find on the basis of count of items}

When we read that the selling price of 25 articles is equal to cost price of 30 articles, we can see that the selling price is definitely greater. We are able to sell 25 items that are worth 30 items, thus making a profit of 5 items. Our profit, therefore, should be calculated from the number of items we are selling

Profit \% = $\dfrac{5}{25} * 100 = \dfrac{1}{5} * 100 = 20\%$

\textbf{Approach 3 : Use count of values}

We know that selling price of 25 articles = cost price of 30 items. Let us assume that cost price of 1 item is Rs 1. 

\begin{itemize}
    \item Cost price of 25 items = Rs 25
    \item Selling price of 25 items = 30 (as provided in question)
    \item Profit obtained by selling 25 items = $30 - 25 = 5$
    \item Profit \% = $\dfrac{5}{25} * 100 = \dfrac{1}{5} * 100 = 20\%$
\end{itemize}

\SampleQuestion{Selling price of 35 bananas = Cost price of 30 bananas. Find loss \%}

Let cost of 1 banana = Rs 1
\begin{itemize}
    \item CP of 35 bananas = Rs 35
    \item SP of 35 bananas = Rs 30 (given)
    \item Loss by selling 35 bananas = 30 - 35 = -5
    \item Loss\% = $\dfrac{-5}{35} * 100 = -14.28\%$
    
\end{itemize}

\SampleQuestion{Selling Price of 27 oranges = Cost price of 36 oranges and discount on 16 oranges = profit on 8 oranges. Find difference between profit \% and discount \%}

Let us assume that CP = 1. 
\begin{itemize}
    \item CP of 27 oranges = 27
    \item SP of 27 oranges = 36 (given)
    \item Profit gained by selling 27 oranges = 36 - 27 = 9
    \item Profit \% = $\dfrac{9}{27} * 100= 33.33\%$
\end{itemize}

We know that profit on 27 oranges = 9. We can say that profit on 8 oranges will be $\dfrac{9}{27} * 8 = \dfrac{8}{3}$. As provided in question, discount on 16 oranges = $\dfrac{8}{3}$. We can also determine the discount on 8 oranges : $\dfrac{8}{3 * 2} = \dfrac{4}{3}$

\begin{itemize}
    \item Discount on 8 oranges = $\dfrac{4}{3}$
    \item Selling price of 8 oranges
    \begin{itemize}
        \item We know that 27 oranges are sold for 36
        \item Therefore, 8 oranges will be sold in $\dfrac{36}{27} * 8 = \dfrac{32}{3}$
        \item \textbf{Discount \% is always calculated on MP}. MP = SP + Discount $\implies MP = \dfrac{32}{3} + \dfrac{4}{3} = \dfrac{36}{3} = 12$
        \item Discount \% = $\dfrac{4}{3 * 12} * 100 = 11.11\%$
    \end{itemize}
\end{itemize}

Difference between profit \% and discount \% = 33.33 - 11.11 = 22.22\%


\SampleQuestion{If selling price is doubled, the profit triples. Find the profit percent}

Let initial selling price be $SP_1$ and initial profit be $P_1$. According to the question, we have two statements
\begin{align}
    SP_1 &= CP + P_1 \\
    2 * SP_1 &= CP + 3 * P_1 \\
    SP_1 &= P_1 \tag{Subtract the above 2 equations} \\
    2P_1 &= CP + P_1 \tag{Substitute value of $SP_1$ in First equation} \\
    CP &= P_1
\end{align}

Profit \% = $\dfrac{P_1}{P_1} * 100 = 100\%$

\SampleQuestion{Selling price of 2 products are same. The profit \% on selling the 1st product is 25\% and the loss on selling the product 2 is 14.28\%. Find overall profit\% or loss\%}

\begin{multicols}{2}
    \textbf{Product 1}
    
    Since profit is 25\%, $SP = \dfrac{5}{4} CP_1 \implies CP_1 = \dfrac{4}{5} SP$
    
    \columnbreak
    \textbf{Product 2}
    
    Since loss is 14.28\%, $SP = \dfrac{6}{7} CP_2 \implies CP_2 = \dfrac{7}{6} SP$
    
\end{multicols}

\begin{itemize}
    \item Total CP = $\dfrac{7}{6}SP + \dfrac{4}{5}SP = \dfrac{35 + 24}{30}SP = \dfrac{59}{30} SP$
    \item Total SP = $2SP$
    \item Delta = $SP - CP = 2SP - \dfrac{59}{30} SP = \dfrac{1}{30}SP$
    \item Profit \% = $\dfrac{\frac{1}{30}}{\frac{59}{30}} SP * 100 = \dfrac{100}{59} = 1.66\%$ (approx)
\end{itemize}

\SampleQuestion{A man buys oranges at the rate of Rs 2 for 6 and sells the whole lot at Rs 3 for 7. What is his profit percentage? And how many oranges must he have purchased in order to make profit of Rs 20?}

To perform any kind of percentage calculation, let us make the quantity same of oranges purchased and sold. We can assume $LCM(6,7) = 42$ oranges. From the above statements
\begin{itemize}
    \item Cost price of 42 oranges = 2 * 7 = 14
    \item Selling price of 42 oranges = 3 * 6 = 18
    \item Profit = 18 - 14 = 4
    \item Profit\% = $\dfrac{4}{14} * 100 = 28.56\%$
\end{itemize}

From above, we know that we get a profit of Rs4 after selling 42 oranges. We now need to calculate the number of oranges we would need to sell to gain a profit of Rs 20
\begin{align*}
    \text{Profit Rs } 4 &= 42 \text{ oranges } \\
    \text{Profit Rs } 1 &= \dfrac{42}{4} \text{ oranges } \\
    \text{Profit Rs } 20 &= \dfrac{42 * 20}{4} \text{ oranges } \\
    &= 210 \text { Oranges }
\end{align*}

\SampleQuestion{By selling 56 toffees a rupee, a vendor loses 30\%. How many toffees for a rupee he must sell to make a profit of 40\% ?}

Let us deal with 1 toffee for now : If 56 toffees are sold at Rs 1, then 1 toffee must be sold at Rs $\dfrac{1}{56}$. The toffees are sold at a loss of 30\% and thus
\begin{align*}
    SP &= 0.7CP \\
    \dfrac{1}{56} &= \dfrac{7}{10} CP \\
    CP &= \dfrac{10}{56 * 7}
\end{align*}

Now, we are supposed to sell $x$ number of toffees at Rs 1 with a profit of 40\%. Therefore, the cost price of $x$ toffees will be 
\begin{align*}
    SP &= 1.4CP_x \\
    1 &= \dfrac{14}{10} CP_x \\
    CP_x &= \dfrac{10}{14}
\end{align*}

Equating using $CP_x$
\begin{align*}
    CP_x &= \dfrac{10}{14} \\
    x * \dfrac{10}{56 * 7} &= \dfrac{10}{14} \\
    x &= \dfrac{56 * 7 * 10}{10 * 14} \\
    &= 28
\end{align*}

\SampleQuestion{The profit made on selling 60m of cloth is equal to SP of 15m cloth. Find profit \%}

\begin{NOTE}
    We are talking about Selling Price, not Cost Price. I made the mistake of assuming cost price when I first attempted the question
\end{NOTE}

Let SP of 1m cloth = Rs 1
\begin{itemize}
    \item SP of 60m cloth = Rs 60
    \item SP of 15m cloth = Rs 15
    \item Profit = SP of 15m cloth = Rs 15
    \item CP = SP - P $\implies CP = 45$
    \item Profit \% = $\dfrac{15}{45} = 33.33\%$
\end{itemize}

\SampleQuestion{A milkman mixes water in his milk. He mixes water, which is 20\% by volume of milk and cost of water is 20\% cost price of milk. If he claims to sell at the cost price, find his profit \%}

\begin{itemize}
    \item Let us assume that 1000 ml of milk is being sold at Rs 1000 $\implies 1 ml = Rs 1$. With this, we can conclude that price of water is Rs 200 per 1000ml $\implies 1 ml = Rs 0.2$.
    
    \item According to the question, milkman mixed 20\% water by volume to already existing milk. That is, in 1000 ml of milk, he mixed 200 ml of water. The cost of this 1200 ml mixture is $1000 * 1 + 200 * 0.2 = 1040$
    
\end{itemize}

The shopkeeper then claims to sell milk at cost price:
\begin{itemize}
    \item CP of 1200 ml milk = Rs 1040
    \item SP of 1200 ml milk = Rs 1200
    \item Profit = Rs 160
    \item Profit \% = $\dfrac{160}{1040} * 100 = 15.4 \% \%$
\end{itemize}

\SampleQuestion{Profit after selling an article for Rs 717 is 11.11\% more than the loss incurred when it is sold at Rs 527. What would be the selling price if he wants to earn a profit of 10\%?}

Let CP be the cost price. ALso, let $SP_1,P_1$ be cost price, selling price and profit in the first scenario where we sell an item for 717 and $SP_2,L_2$ be cost price, selling price and loss in the second scenario where we sell an item for 527. According to the question

\begin{itemize}
    \item $SP_1 = CP + P_1$
    \item $P_1 = (1 + \dfrac{1}{9} ) * L_2$ as Profit is 11.11\% more than the loss incurred in second scenario
    \item $L_1 = CP - SP_2$
    \item Therefore, $SP_1 = CP + (1 + \dfrac{1}{9}) * (CP - SP_2)$
\end{itemize}

\begin{align*}
    SP_1 &= CP + (1 + \dfrac{1}{9}) * (CP - SP_2) \\
    717 &= CP + (\dfrac{10}{9} * (CP - 527)) \\
    9 * (717 - CP) &= 10CP - 5270 \\
    6300 + 90 + 63 - 9CP &= 10CP - 5270 \\
    19CP &= 6300 + 5200 + 70 + 90 + 63 \\
    CP &= \dfrac{11723}{19}
    &= 617
\end{align*}

Selling price at a 10\% profit = 617 + 61.7 = 678.1

\section{Miscellaneous Questions}

\SampleQuestion{A shopkeeper bought an article at Rs 1000 and marked up price by $x\%$. If he then gave a discount of $\dfrac{2x}{5} \%$ and still got a profit of $\dfrac{2x}{5} \%$, find the amount of markup percentage}

\begin{multicols}{2}
    \begin{enumerate}
        \item 40
        \item 50
    \end{enumerate}

    \columnbreak
    
    \begin{enumerate} \addtocounter{enumi}{2}
        \item 60
        \item 70
    \end{enumerate}
    
\end{multicols}

Finding marked price
\begin{align*}
    MP &= ( 1 + \dfrac{x}{100} ) CP \\
    &= \dfrac{100 + x}{100} * 1000 \\
    &= 1000 + 10x
\end{align*}

Finding selling price through the discount offered on marked price
\begin{align*}
    SP &= (1 - \dfrac{2x}{500}) * MP \\
    &= \dfrac{500 - 2x}{500} * (1000 + 10x)
\end{align*}

Finding selling price through the profit earned on cost price
\begin{align*}
    SP &= (1 + \dfrac{2x}{500}) * CP \\
    &= \dfrac{500 + 2x}{500} * 1000 \\
    &= 1000 + 4x
\end{align*}

Equating the selling prices we found above
\begin{align*}
    1000 + 4x &= \dfrac{500 - 2x}{500} * (1000 + 10x) \\
    500 * (1000 + 4x) &= (500 - 2x) * 10 * (100 + x) \\
    50000 + 200x &= 50000 + 500x - 200x - 2x^2 \\
    2x^2 &= 100x \\
    x^2 - 50x &= 0 \\
    x (x - 50) &= 0
    x &= 50
\end{align*}

Answer = 50\%

\begin{EXTRA-LEARNING}
    We can also use options : Arrange the options in ascending order and substitute middle values. We can arrange the options in ascending order like $40,50,60,70$. Then, use the middle values to substitute value of $x$ that is, try with both 50 and 60.  
\end{EXTRA-LEARNING}

\SampleQuestion{A trader sells cakes in economy packs of 4 cakes per pack with each pack being charged at the listed price of 3 cakes. For every set of 5 such cakes bought by a customer, trader gives him one extra cake as a free gift. If customer buys 12 economy packs, what is the effective percentage of discount that he gets?}

Let price of 1 cake be Rs 100. Using the above data
\begin{itemize}
    \item 1 pack contains 4 cakes and is charged at price of 3 cakes $\implies $ Rs 300
    \item For every 5 packs bought, we get 1 cake free 
    \item Customer bought 12 packs and thus, got $\floor{\dfrac{12}{10}} = 2$ cakes free
    \item Total cakes customer got = 12 * 4 + 2 = 50 at a price of 12 * 300 = 3600
\end{itemize}

Now, let us compare : Selling price of 50 cakes = 3600 and cost price of 50 cakes = 5000. Discount = 3600 - 5000 = -1400. Discount \% = $\dfrac{1400}{5000} * 100 = 28\%$

\SampleQuestion{If the selling price of ten oranges is equal to cost price of 14 oranges, which, in turn, is equal to $\frac{1}{3}$ of total discount offered upon 70 oranges, then find the profit / loss percentage when the markup percentage is halved and discount percentage is decreased by 5 percentage points}

\begin{NOTE}
    A percentage point is 1\%. If 65\%  something is reduced by 5 percentage points, then it is now equal to 60\%
\end{NOTE}

Let CP of 1 orange = 100. We are given that 10 oranges are sold at selling price of 14 oranges
\begin{itemize}
    \item CP of 10 oranges = 1000
    \item SP of 10 oranges = 1400
    \item Profit\% = $\dfrac{1400 - 1000}{1000} * 100 = 40\%$
\end{itemize}

Now, we are also given that SP of 10 oranges = $\dfrac{1}{3}$ discount on 70 oranges $\implies $ Discount = $3 * 1400 = 4200$. On 70 oranges
\begin{itemize}
    \item CP of 70 oranges = 7000
    \item SP of 70 oranges = 1.4 * 7000 = 9800 (as profit\% is 40\%)
    \item Discount on 70 oranges = 4200
    \item MP of 70 oranges = 9800 + 4200 = 14000
    \item Discount\% = $\dfrac{4200}{14000} * 100 = 30\%$
    \item Markup percentage = $\dfrac{14000 - 7000}{7000} * 100 = 100\%$
\end{itemize}

We need to find profit when markup is reduced by half \% and discount is reduced by 5 percentage points
\begin{itemize}
    \item CP of 1 orange = 100
    \item MP of 1 orange = 150 (markup percentage = 50\%)
    \item Discount on 1 orange = $\dfrac{1}{4} * 150 = 37.5$ (30 - 5 = 25\% discount)
    \item SP of 1 orange = 150 - 37.5 = 112.5
    \item Profit \% = $\dfrac{112.5 - 100}{100} * 100 = 12.5\%$
\end{itemize}
\newpage



















\section{Cheating and Faulty Weights}
In this section, we will look at types of questions where the shopkeeper is cheating us by adjusting the weight and price. See the following questions

\SampleQuestion{Shopkeeper is using a weight of 800gm instead of 1Kg rate and is claiming to sell at cost price. Find profit \% }

\begin{multicols}{2}
    \textbf{Method 1}
    
    Let us assume that CP of 1gm is Rs 1. According to this
    \begin{itemize}
        \item CP of 800gm = Rs 800
        \item SP of 800gm = Rs 1000 (as the shopkeeper claims to sell 1Kg at cost price)
        \item Profit \% = $\dfrac{1000 - 800}{800} * 100 = 25\% $
    \end{itemize}
    
    \columnbreak

    \textbf{Method 2}
    
    We can compare the quantity of item sold : 800gm of something is being sold at the price of 1000gm. \\
    
    We are selling 200gm more $\implies \dfrac{200}{800} * 100 = 25\% $ profit
\end{multicols}

\SampleQuestion{Shopkeeper is using a weight of 800gm instead of 1Kg rate and is selling after marking the cost price up 40\%. Find the profit \%}

\begin{multicols}{2}
    \textbf{Method 1}
    
    Let us assume that CP of 1gm is Rs 1. According to this
    \begin{itemize}
        \item CP of 800gm = Rs 800
        \item SP of 800gm = Rs 1400 (as the shopkeeper is marking up the price of 1Kg at 40\%)
        \item Profit \% = $\dfrac{1400 - 800}{800} * 100 = 75\% $
    \end{itemize}
    
    \columnbreak

    \textbf{Method 2}
    
    We can compare the quantity of item sold : 800gm of something is being sold at the price of 1000gm. \\
    
    We are selling 200gm more $\implies \dfrac{200}{800} * 100 = 25\% $ profit \\

    Let us take an arbitrary value $100x$ and see how the percentage increases by shopkeeper affect the price
    \begin{itemize}
        \item 25\% profit made by shopkeeper by selling 800gm : $1.25 * 100x = 125x$
        \item 40\% markup made by shopkeeper on cost price : $1.4 * 125x = 175x$
        \item Overall profit \% = $\dfrac{175x - 100x}{100x} * 100 = 75\%$
    \end{itemize}
\end{multicols}

\begin{EXTRA-LEARNING}
    We can extend the method 2 above to include more increases / decreases as well : Suppose the shopkeeper in above question now offers a discount of 40\% after marking up. We found the change to be $175x$. \\
    
    Applying 40\% discount will result in $0.6 * 175x = 105x$. Profit\% = $\dfrac{105x - 100x}{100} * 100 = 5\%$
\end{EXTRA-LEARNING}